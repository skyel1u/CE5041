\begin{center}
    \section{遇到的坑}
\end{center}

\setlength{\parindent}{2em}
\subsection{jemalloc的安装}
jemalloc自己实现了对glibc中的malloc函数的hook操作,于是我们可以直接下载jemalloc,解压至指定目录,使用参数\Verb+./configure --prefix=/path --enable-debug+就可以得到有调试符号的jemalloc了。

然后我们就可以通过在\Verb+ /etc/ld.so.conf+中添加编译好jemalloc的位置,保存之后运行命令\Verb+ ldconfig+,编译程序的时候就可以使用动态链接库的方式链接了(如下代码第3、4行所示):
\begin{minted}[breaklines, frame=lines, linenos]{makefile}
PROGRAMS = fastbin_dup fastbin_dup_into_stack fastbin_dup_consolidate unsafe_unlink house_of_spirit poison_null_byte malloc_playground first_fit house_of_lore overlapping_chunks overlapping_chunks_2 house_of_force unsorted_bin_attack house_of_einherjar house_of_orange
CFLAGS += -std=c99 -g3
LDFLAGS += -L/usr/local/jemalloc/lib
LDLIBS += -ljemalloc

all: $(PROGRAMS)
clean:
    rm -f $(PROGRAMS)
\end{minted}

\subsection{在Ubuntu 16.04下使用高版本Glibc}
Ubuntu自带的glibc版本是2.23, 我们在不替换原系统glib的情况下,若要使用高版本glibc当作共享链接库,则\textbf{万万不可像jemalloc那样运行\Verb+ ldconfig+}。

原因是,glibc并不单纯的包含libc.so.2,而是包含了该对应版本的链接器等一系列工具链,如果我们此时还在使用旧版本的glibc,而运行了\Verb+ ldconfig+之后,就会造成系统的共享链接库错误,此时唯一的解决办法是重新安装系统。因此如果使用高版本的glibc共享库的话,我们就需要指定额外的参数传递给gcc编译器: 
\begin{minted}[breaklines, frame=lines]{bash}
$ gcc -g3 -L/root/tmpwork/g227/lib -Wl,--rpath=/root/tmpwork/g227/lib -Wl,--dynamic-linker=/root/tmpwork/g227/lib/ld-linux-x86-64.so.2 house_of_einherjar.c -o house_of_einherjar

$ ldd house_of_einherjar                                                                                                                                                   
    linux-vdso.so.1 =>  (0x00007ffe979ef000)
    libc.so.6 => /root/tmpwork/g227/lib/libc.so.6 (0x00007fb64da78000)
    /root/tmpwork/g227/lib/ld-linux-x86-64.so.2 => /lib64/ld-linux-x86-64.so.2 (0x00007fb64de2d000)

\end{minted}

其中,\Verb+ -Wl+指的是将该参数传递给链接器,我们在编译时指定了所使用的动态链接库,以及和该版本动态链接库对应的链接器。这样,程序就可以使用高版本的glibc了。

\newpage