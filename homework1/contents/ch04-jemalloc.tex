\begin{center}
    \section{jemalloc 分析}
\end{center}

\setlength{\parindent}{2em}
此处分析的版本为jemalloc-5.0.1中的\verb+ malloc+实现。
为了方便调试,我们可以在编译\verb+ jemalloc+的时候保留调试符号信息:\verb+ --enable-debug+。
修改\verb+ Makefile+,使用jemalloc编译how2heap的二进制文件:
\begin{minted}[breaklines, frame=lines]{bash}
PROGRAMS = fastbin_dup fastbin_dup_into_stack fastbin_dup_consolidate unsafe_unlink house_of_spirit poison_null_byte malloc_playground first_fit house_of_lore overlapping_chunks overlapping_chunks_2 house_of_force unsorted_bin_attack house_of_einherjar house_of_orange
CFLAGS += -std=c99 -g
LDFLAGS += -L/usr/local/jemalloc/lib
LDLIBS += -ljemalloc
# Convenience to auto-call mcheck before the first malloc()
#CFLAGS += -lmcheck

all: $(PROGRAMS)
clean:
    rm -f $(PROGRAMS)
\end{minted}

\subsection{jemalloc简介}

现代CPU已大多数为多核CPU,多线程的应用程序也越来越广泛,内存的分配与回收也越来越成为制约程序性能的一大原因。因此为多核多线程的堆管理器jemalloc应运而生。

在过去,分配器使用\verb+ sbrk(2)+来获得内存,由于多种原因,其中包括竞争条件,碎片增加以及最大可用内存的人为限制,这是不理想的。 如果操作系统支持\verb+ sbrk(2)+,则该分配器按照该优先顺序同时使用\verb+ mmap(2)+和\verb+ sbrk(2)+; 否则仅使用\verb+ mmap(2)+。

这个堆分配器会使用多arena的方式来减少多核系统上多线程程序的锁竞争。除了多个arena之外,该分配器还支持线程特定的缓存,以便可以完全避免大多数分配请求的同步。 这样的缓存允许在正常情况下进行快速分配,但是它增加了内存的使用和分段,因为有限数量的对象可以在每个线程缓存中一直保存被分配的状态。

从jemalloc 5.0.0版本开始,不再使用"chunks"这个数据结构进行虚拟内存的管理,而是去使用一个新的,页面对齐的数据结构——extents。(于是之前几乎所有关于jemalooc的资料全部作废了,就是一夜回到解放前看源码)

在jemalloc的最新设计中,内存区域在概念上被划分为extents,extents始终与页面大小的倍数对齐,这种对齐可以让用户更快速的找到储存的元数据。用户的对象被分为大小两类,连续的小对象包含一个slab,也就是一个单一的extent,而大的对象则有自己的extents支持。

小的对象由slab分组管理,而每一个slab则维护一个bitmap追踪哪些区域正在被使用。不超过1 quantum一半的分配请求(8或16,依据架构而定)将四舍五入到最接近的2的幂,至少是\verb+ sizeof(double)+。

具体大小和分类见下表 \footnote{详见:http://jemalloc.net/jemalloc.3.html\#size\_classes}:
% Please add the following required packages to your document preamble:
\begin{table}[]
\centering
\begin{tabular}{|l|l|l|}
\hline
\multicolumn{1}{|c|}{Category} & Spacing & Size                                   \\ \hline
\multirow{9}{*}{Small}               & lg      & {[}8{]}                                \\ \cline{2-3} 
                               & 16      & {[}16, 32, 48, 64, 80, 96, 112, 128{]} \\ \cline{2-3} 
                               & 32      & {[}160, 192, 224, 256{]}               \\ \cline{2-3} 
                               & 64      & {[}320, 384, 448, 512{]}               \\ \cline{2-3} 
                               & 128     & {[}640, 768, 896, 1024{]}              \\ \cline{2-3} 
                               & 256     & {[}1280, 1536, 1792, 2048{]}           \\ \cline{2-3} 
                               & 512     & {[}2560, 3072, 3584, 4096{]}           \\ \cline{2-3} 
                               & 1KiB    & {[}5 KiB, 6 KiB, 7 KiB, 8 KiB{]}       \\ \cline{2-3} 
                               & 2KiB    & {[}10 KiB, 12 KiB, 14 KiB{]}           \\ \hline
\multirow{3}{*}{Large}               & 2 KiB   & {[}16 KiB{]}                           \\ \cline{2-3} 
                               & ...     & ..                                     \\ \cline{2-3} 
                               & 1 EiB   & {[}5 EiB, 6 EiB, 7 EiB{]}              \\ \hline
\end{tabular}
\end{table}


\subsection{unsafe\_unlink}

由于jemalloc的实现和ptmalloc并不相同,没有采用边界标记来在堆分配的时候标记数据,而是使用了bitmap标记内存的使用情况,因此jemalloc对unlink攻击免疫,所以unlink攻击无法成功。
\begin{minted}[breaklines, frame=lines]{bash}
$ how2heap ./unsafe_unlink
The global chunk0_ptr is at 0x601070, pointing to 0x7ff94e61e000
The victim chunk we are going to corrupt is at 0x7ff94e61e080

Fake chunk fd: 0x601058
Fake chunk bk: 0x601060

Original value: AAAAAAAA
New Value: AAAAAAAA

\end{minted}

\subsection{house\_of\_spirit}

攻击失败,运行结果如下:
\begin{minted}[breaklines, frame=lines]{bash}
$ how2heap ./house_of_spirit
We will overwrite a pointer to point to a fake 'fastbin' region. This region contains two chunks.
The first one:  0x7ffd67374070
The second one: 0x7ffd67374090
Overwritting our pointer with the address of the fake region inside the fake first chunk, 0x7ffd67374070.
Freeing the overwritten pointer.
<jemalloc>: src/rtree.c:205: Failed assertion: "!dependent || leaf != NULL"
[1]    27382 abort      ./house_of_spirit

\end{minted}

free时会调用jemalloc中的\verb+ free()+,跟踪进入\verb+ free()+调用的\verb+ ifree()+:
\begin{minted}[breaklines, frame=lines]{c}
ifree(tsd=0x7ffff7fde728, ptr=0x7fffffffe480, tcache=0x7ffff7fde8e0, slow_path=0x1);
\end{minted}

继续跟入,\verb+ tsd_tsdn()+函数十分简洁,且作用是强制类型转换,我们可以略过,进入\verb+ ifree()+中调用的\verb+ idalloctm()+函数,再进入\verb+ arena_dalloc()+:
\begin{minted}[breaklines, frame=lines]{c}
// arena_dalloc(tsdn=0x7ffff7fde728, ptr=0x7fffffffe480, tcache=0x7ffff7fde8e0, alloc_ctx=0x7fffffffe3f0, slow_path=0x1)
    // config_debug = 1, 进入
    if (config_debug) {
        rtree_ctx = tsd_rtree_ctx(tsdn_tsd(tsdn));
        // extent=0x00007fffffffe348  →  [...]  →  0x0000000100000101, tsdn=0x00007fffffffe328  →  [...]  →  0x0000000100000101
        extent_t *extent = rtree_extent_read(tsdn, &extents_rtree,
            rtree_ctx, (uintptr_t)ptr, true);
        // extent=0x00007fffffffe348  →  0x0000000000000000
        assert(szind == extent_szind_get(extent));
        assert(szind < NSIZES);
        assert(slab == extent_slab_get(extent));
    }
    // 略去无关函数
\end{minted}

我们在\verb+ extent_szind_get(extent)+函数中调用的\verb+ extent_szind_get_maybe_invalid(extent)+中有如下操作:
\begin{minted}[breaklines, frame=lines]{c}
    static inline szind_t
        extent_szind_get_maybe_invalid(const extent_t *extent) {
        // extent=0x00007fffffffe2a8  →  0x0000000000000000
        szind_t szind = (szind_t)((extent->e_bits & EXTENT_BITS_SZIND_MASK) >>
            EXTENT_BITS_SZIND_SHIFT);
        assert(szind <= NSIZES);
        return szind;
}

\end{minted}

extent的值由\verb+ 0x00007fffffffe2a8+变成了一个非法地址\verb+ 0x0000000000000000+,因此\verb+ extent->e_bits+无法计算,所以,\verb+ szind+的值与\verb+ NSIZES+(该值此时为0xE8)无法比较,因此程序收到终止信号,程序退出。

\subsection{house\_of\_lore}

攻击失败:
\begin{minted}[breaklines, frame=lines]{bash}
$ how2heap ./house_of_lore 
Allocated the victim (small) chunk: 0x7fd6d8878000
stack_buffer_1: 0x7ffd6419e2a0
stack_buffer_2: 0x7ffd6419e280

Freeing the victim chunk 0x7fd6d8878000, it will be inserted in the unsorted bin
victim->fd: 0x5a5a5a5a5a5a5a5a
victim->bk: 0x5a5a5a5a5a5a5a5a

Malloc a chunk that can't be handled by the unsorted bin, nor the SmallBin: 0x7fd6d887a000
The victim chunk 0x7fd6d8878000 will be inserted in front of the SmallBin
victim->fd: 0x5a5a5a5a5a5a5a5a
victim->bk: 0x5a5a5a5a5a5a5a5a

Now emulating a vulnerability that can overwrite the victim->bk pointer
This last malloc should return a chunk at the position injected in bin->bk: 0x7fd6d8878000
The fd pointer of stack_buffer_2 has changed: 0x7ffd6419e2a0

\end{minted}

gdb打开应用,在free之前下断点,堆布局是这样的:
\begin{minted}[breaklines, frame=lines]{c}
gef> x/20gx victim-2
0x7ffff64a4ff0: 0x0000000000000000  0x0000000000000000
0x7ffff64a5000: 0x4141414141414141  0x4141414141414141
...           : ...                 ..
0x7ffff64a5070: 0x4141414141414141  0x4141414141414141
0x7ffff64a5080: 0x0000000000000000  0x0000000000000000
\end{minted}

之前就已经提到过,jemalloc没有采用边界标记法表示内存,因此没有在分配内存的附近有标记,如果使用glibc的堆布局就如下所示:
\begin{minted}[breaklines, frame=lines]{c}
gef> x/20gx victim-2
0x603000:   0x0000000000000000  0x0000000000000091 <- size
0x603010:   0x4141414141414141  0x4141414141414141
...     :   ...                 ...
0x603080:   0x4141414141414141  0x4141414141414141
0x603090:   0x0000000000000000  0x0000000000000021 <- next size
\end{minted}

\verb+ free()+之后,我们的堆布局变成了下面的样子:
\begin{minted}[breaklines, frame=lines]{bash}
gef> x/20gx victim-2
0x7ffff64a4ff0: 0x0000000000000000  0x0000000000000000
0x7ffff64a5000: 0x5a5a5a5a5a5a5a5a  0x5a5a5a5a5a5a5a5a
...           : ...                 ...
0x7ffff64a5070: 0x5a5a5a5a5a5a5a5a  0x5a5a5a5a5a5a5a5a
0x7ffff64a5080: 0x0000000000000000  0x0000000000000000
\end{minted}

jemalloc在free某块内存之后,不仅会在bitmap中标记,还会填充特定的字符,因此攻击失效。

\subsection{house\_of\_einherjar}

前面的分析已经提到过,house-of-einherjar 是一种利用 malloc 来返回一个附近地址的任意指针。它要求有一个单字节溢出漏洞,覆盖掉\verb+ next chunk+的 \verb+ size+字段并清除\verb+ PREV_IN_USE+标志。jemalloc没有采用边界标记的方法标记\verb+ free chunks+,因此本次攻击也是不成功的。
\begin{minted}[breaklines, frame=lines]{bash}
$ how2heap ./house_of_einherjar
We allocate 0x10 bytes for 'a': 0x7f6121abf000

Our fake chunk at 0x7ffe5fe58050 looks like:
prev_size: 0x80
size: 0x80
fwd: 0x7ffe5fe58050
bck: 0x7ffe5fe58050
fwd_nextsize: 0x7ffe5fe58050
bck_nextsize: 0x7ffe5fe58050

We allocate 0xf8 bytes for 'b': 0x7f6121ac0000
b.size: 0
We overflow 'a' with a single null byte into the metadata of 'b'
b.size: 0

We write a fake prev_size to the last 8 bytes of a so that it will consolidate with our fake chunk
Our fake prev_size will be 0x7f6121abfff0 - 0x7ffe5fe58050 = 0xffffff62c1c67fa0

Modify fake chunk's size to reflect b's new prev_size
Now we free b and this will consolidate with our fake chunk
Our fake chunk size is now 0xffffff62c1c67fa0 (b.size + fake_prev_size)

Now we can call malloc() and it will begin in our fake chunk: 0x7f6121abf010
\end{minted}

在分配\verb+ b+时下断点,查看堆布局:
\begin{minted}[breaklines, frame=lines]{bash}
gef$ x/50gx b-0x10
0x7ffff64a5ff0: 0x0000000000000000  0x0000000000000000
0x7ffff64a6000: 0xa5a5a5a5a5a5a5a5  0xa5a5a5a5a5a5a5a5
...           : ...                 ...
0x7ffff64a60f0: 0xa5a5a5a5a5a5a5a5  0xa5a5a5a5a5a5a5a5
0x7ffff64a6100: 0x0000000000000000  0x0000000000000000
\end{minted}

而使用glibc分配的内存如下所示:
\begin{minted}[breaklines, frame=lines]{bash}
gef$ x/50gx b-0x10
0x603020:   0x4141414141414141  0x0000000000000101 <- P flag
0x603030:   0x0000000000000000  0x0000000000000000 -
...     :   ...                 ...                | <- chunk b
0x603120:   0x0000000000000000  0x0000000000020ee1 -
0x603130:   0x0000000000000000  0x0000000000000000
\end{minted}

综上,本次攻击也是失败的。

\subsection{house\_of\_orange}

运行程序没有任何输出,自然是攻击失败:
\begin{minted}[breaklines, frame=lines]{bash}
$ how2heap ./house_of_orange
$ how2heap 

\end{minted}

前文提到,house\_of\_orange的攻击是利用\verb+ unsorted_bins+的\verb+ unlink+伪造 \verb+ _IO_jump_t+中的\verb+ __overflo+为\verb+ system()+函数的地址,从而达到执行 shell 的目的;而jemalloc中既没有\verb+ unsorted_bins+,也没有\verb+ unlink+,因此攻击失败。

\subsection{house\_of\_force}

攻击失败:

\begin{minted}[breaklines, frame=lines]{bash}
$ how2heap ./house_of_force 
We will overwrite a variable at 0x602080

Let's allocate the first chunk of 0x10 bytes: 0x7f35be7cf000.
Real size of our allocated chunk is 0x10.

Overwriting the top chunk size with a big value so the malloc will never call mmap.
Old size of top chunk: 0
New size of top chunk: 0xffffffffffffffff

The value we want to write to at 0x602080, and the top chunk is at 0x7f35be7cf010, so accounting for the header size, we will malloc 0xffff80ca41e33060 bytes.
[1]    12108 segmentation fault  ./house_of_force

\end{minted}

前文已经说过,house\_of\_lore 的原理是通过破坏已经放入small bins 中的\verb+ bk+ 指针来达到取得任意地址的目的,jemalloc没有采用链表的结构管理bins,自然攻击失败。

\subsection{poison\_null\_byte}

\subsection{unsorted\_bin\_attack}

不成功,jemalloc中既没有\verb+ unsorted_bins+这个数据结构,也不是通过链表管理bins:
\begin{minted}[breaklines, frame=lines]{bash}
$./unsorted_bin_attack 
This file demonstrates unsorted bin attack by write a large unsigned long value into stack
In practice, unsorted bin attack is generally prepared for further attacks, such as rewriting the global variable global_max_fast in libc for further fastbin attack

Let's first look at the target we want to rewrite on stack:
0x7ffc82427168: 0

Now, we allocate first normal chunk on the heap at: 0x7fd534473000
And allocate another normal chunk in order to avoid consolidating the top chunk withthe first one during the free()

We free the first chunk now and it will be inserted in the unsorted bin with its bk pointer point to 0x5a5a5a5a5a5a5a5a
Now emulating a vulnerability that can overwrite the victim->bk pointer
And we write it with the target address-16 (in 32-bits machine, it should be target address-8):0x7ffc82427158

Let's malloc again to get the chunk we just free. During this time, target should has already been rewrite:
0x7ffc82427168: (nil)

\end{minted}

\subsection{unsorted\_bin\_into\_stack}

\subsection{fastbin\_dup}

编译,运行:
\begin{minted}[breaklines, frame=lines]{bash}
$ ./fastbin_dup
This file demonstrates a simple double-free attack with fastbins.
Allocating 3 buffers.
1st malloc(8): 0x7f301d3a6008
2nd malloc(8): 0x7f301d3a6010
3rd malloc(8): 0x7f301d3a6018
Freeing the first one...
If we free 0x7f301d3a6008 again, things will crash because 0x7f301d3a6008 is at the top of the free list.
So, instead, we'll free 0x7f301d3a6010.
Now, we can free 0x7f301d3a6008 again, since it's not the head of the free list.
Now the free list has [ 0x7f301d3a6008, 0x7f301d3a6010, 0x7f301d3a6008 ]. If we malloc 3 times, we'll get 0x7f301d3a6008 twice!
1st malloc(8): 0x7f301d3a6008
2nd malloc(8): 0x7f301d3a6010
3rd malloc(8): 0x7f301d3a6008

\end{minted}

看上去结果和Glibc的类似,实际上内部完全不同。前文已经提到,对于小块内存,jemalloc使用slab来管理。

可以猜测对于小块内存,je如果在使用tcache的情况下,采用了一种栈的数据结构结构来管理,写demo,先malloc三次,再free三次,再malloc,那么我们得到的还是之前的内存的地址:
\begin{minted}[breaklines, frame=lines]{c}
#include <stdio.h>
#include <stdlib.h>
#include <string.h>

int main(){
    long *a, *b, *c;
    a = malloc(8);
    b = malloc(8);
    c = malloc(8);
    printf("a: %#x\n", a);
    printf("b: %#x\n", b);
    printf("c: %#x\n", c);
    free(a);
    free(b);
    free(c);
    a = malloc(8);
    b = malloc(8);
    c = malloc(8);
    printf("a: %#x\n", a);
    printf("b: %#x\n", b);
    printf("c: %#x\n", c);

    return 0;

}

\end{minted}

运行得到结果:
\begin{minted}[breaklines, frame=lines]{bash}
$ ./haha       
a: 0xe8d64008
b: 0xe8d64010
c: 0xe8d64018
a: 0xe8d64018
b: 0xe8d64010
c: 0xe8d64008

\end{minted}

可以看出是一个先入后出的结构,实际上在jemalloc源码中也能看到有关此结构的定义:
\begin{minted}[breaklines, frame=lines]{c}
// tcache_s
struct tcache_s
{
    /* Data accessed frequently first: prof, ticker and small bins. */
    uint64_t prof_accumbytes; /* Cleared after arena_prof_accum(). */
    ticker_t gc_ticker;       /* Drives incremental GC. */
    /*
     * The pointer stacks associated with tbins follow as a contiguous
     * array.  During tcache initialization, the avail pointer in each
     * element of tbins is initialized to point to the proper offset within
     * this array.
     */
    tcache_bin_t tbins_small[NBINS];
    /* Data accessed less often below. */
    ql_elm(tcache_t) link; /* Used for aggregating stats. */
    arena_t *arena;        /* Associated arena. */
    szind_t next_gc_bin;   /* Next bin to GC. */
    /* For small bins, fill (ncached_max >> lg_fill_div). */
    uint8_t lg_fill_div[NBINS];
    tcache_bin_t tbins_large[NSIZES - NBINS];
};

// tcache_bin_s
struct tcache_bin_s
{
      low_water_t low_water; /* Min # cached since last GC. */
      uint32_t ncached; /* # of cached objects. */
      /*
     * ncached and stats are both modified frequently.  Let's keep them
     * close so that they have a higher chance of being on the same
     * cacheline, thus less write-backs.
     */
      tcache_bin_stats_t tstats;
      /*
     * To make use of adjacent cacheline prefetch, the items in the avail
     * stack goes to higher address for newer allocations.  avail points
     * just above the available space, which means that
     * avail[-ncached, ... -1] are available items and the lowest item will
     * be allocated first.
     */
      void **avail; /* Stack of available objects. */
};
\end{minted}

实际上我们可以画一个简单的图表示:
\begin{minted}[breaklines, frame=lines]{bash}
            +---------------------+
          / | link                |
tcache_t <  | prof_accumbytes     |
          | | gc_ticker           |
          \ | next_gc_bin         |
            |---------------------|
          / | tstats              |
          | | low_water           |
tbins[0] <  | lg_fill_div         |
          | | ncached             |
          \ | avail               |--+
            |---------------------|  |
          / | tstats              |  |
          | | low_water           |  |
          | | lg_fill_div         |  |                     Run
tbins[1] <  | ncached             |  |                +-----------+
          | | avail               |--|--+             |  region   |
          \ |---------------------|  |  |             |-----------|
            ...  ...  ...            |  |   +-------->|  region   |
            |---------------------|  |  |   |         |-----------|
            | padding             |  |  |   |         |  region   |
            |---------------------|<-+  |   |         |-----------|
            | stack[0]            |-----|---+   +---->|  region   |
            | stack[1]            |-----|-------+     |-----------|
            | ...                 |     |             |  region   |
            | stack[ncache_max-1] |     |             |-----------|
            |---------------------|<----+      +----->|  region   |
            | stack[0]            |------------+      |-----------|
            | stack[1]            |                   |           |
            | ...                 |                   |           |
            | stack[ncache_max-1] |                   |           |
            |---------------------|                   |           |
            ...  ...  ...                             ...  ...  ...
            +---------------------+                   +-----------+
\end{minted}

\subsection{fastbin\_dup\_into\_stack}

看上去结果和glibc的类似,实际上攻击是不成功的:
\begin{minted}[breaklines, frame=lines]{bash}
$ ./fastbin_dup_into_stack 
This file extends on fastbin_dup.c by tricking malloc into
returning a pointer to a controlled location (in this case, the stack).
The address we want malloc() to return is 0x7ffe12338c88.
Allocating 3 buffers.
1st malloc(8): 0x7f979bad3008
2nd malloc(8): 0x7f979bad3010
3rd malloc(8): 0x7f979bad3018
Freeing the first one...
If we free 0x7f979bad3008 again, things will crash because 0x7f979bad3008 is at the top of the free list.
So, instead, we'll free 0x7f979bad3010.
Now, we can free 0x7f979bad3008 again, since it's not the head of the free list.
Now the free list has [ 0x7f979bad3008, 0x7f979bad3010, 0x7f979bad3008 ]. We'll now carry out our attack by modifying data at 0x7f979bad3008.
1st malloc(8): 0x7f979bad3008
2nd malloc(8): 0x7f979bad3010
Now the free list has [ 0x7f979bad3008 ].
Now, we have access to 0x7f979bad3008 while it remains at the head of the free list.
so now we are writing a fake free size (in this case, 0x20) to the stack,
so that malloc will think there is a free chunk there and agree to
return a pointer to it.
Now, we overwrite the first 8 bytes of the data at 0x7f979bad3008 to point right before the 0x20.
3rd malloc(8): 0x7f979bad3008, putting the stack address on the free list
4th malloc(8): 0x7f979bad3020

\end{minted}

在上一个攻击:fastbin\_dup中我们已经分析过,这里就不赘述了。

\subsection{overlapping\_chunks}

攻击失败,前面提到,overlapping\_chunks的攻击方法是改写\verb+ unsorted bin +中空闲堆块的 size,改变下一次\verb+ malloc()+可以返回的堆块大小,jemalloc未采用边界标记的方法标记内存,自然攻击不成功:
\begin{minted}[breaklines, frame=lines]{bash}
$ ./overlapping_chunks 

This is a simple chunks overlapping problem

Let's start to allocate 3 chunks on the heap
The 3 chunks have been allocated here:
p1=0x7f2e4b5af000
p2=0x7f2e4b5af100
p3=0x7f2e4b5b0000

Now let's free the chunk p2
The chunk p2 is now in the unsorted bin ready to serve possible
new malloc() of its size
Now let's simulate an overflow that can overwrite the size of the
chunk freed p2.
For a toy program, the value of the last 3 bits is unimportant; however, it is best to maintain the stability of the heap.
To achieve this stability we will mark the least signifigant bit as 1 (prev_inuse), to assure that p1 is not mistaken for a free chunk.
We are going to set the size of chunk p2 to to 385, which gives us
a region size of 376

Now let's allocate another chunk with a size equal to the data
size of the chunk p2 injected size
This malloc will be served from the previously freed chunk that
is parked in the unsorted bin which size has been modified by us

p4 has been allocated at 0x7f2e4b5b1000 and ends at 0x7f2e4b5b1bc0
p3 starts at 0x7f2e4b5b0000 and ends at 0x7f2e4b5b0280
p4 should overlap with p3, in this case p4 includes all p3.

Now everything copied inside chunk p4 can overwrites data on
chunk p3, and data written to chunk p3 can overwrite data
stored in the p4 chunk.

Let's run through an example. Right now, we have:
p4 = ¥¥¥¥¥¥¥¥¥¥¥¥¥¥¥¥¥¥¥¥¥¥¥¥¥¥¥¥¥¥¥¥¥¥¥¥¥¥¥¥¥¥¥¥¥¥¥¥¥¥¥¥¥¥¥¥¥¥¥¥¥¥
¥¥¥¥¥¥¥¥¥¥¥¥¥¥¥¥¥¥¥¥¥¥¥¥¥¥¥¥¥¥¥¥¥¥¥¥¥¥¥¥¥¥¥¥¥¥¥¥¥¥¥¥¥¥¥¥¥¥¥¥¥¥¥¥¥¥¥
¥¥¥¥¥¥¥¥¥¥¥¥¥¥¥¥¥¥¥¥¥¥¥¥¥¥¥¥¥¥¥¥¥¥¥¥¥¥¥¥¥¥¥¥¥¥¥¥¥¥¥¥¥¥¥¥¥¥¥¥¥¥¥¥¥¥¥
¥¥¥¥¥¥¥¥¥¥¥¥¥¥¥¥¥¥¥¥¥¥¥¥¥¥¥¥¥¥¥¥¥¥¥¥¥¥¥¥¥¥¥¥¥¥¥¥¥¥¥¥¥¥¥¥¥¥¥¥¥¥¥¥¥¥¥
¥¥¥¥¥¥¥¥¥¥¥¥¥¥¥¥¥¥¥¥¥¥¥¥¥¥¥¥¥¥¥¥¥¥¥¥¥¥¥¥¥¥¥¥¥¥¥¥¥¥¥¥¥¥¥¥¥¥¥¥¥¥¥¥¥¥¥
¥¥¥¥¥¥¥¥¥¥¥¥¥¥¥¥¥¥¥¥¥¥¥¥¥¥¥¥¥¥¥¥¥¥¥¥¥¥¥¥¥¥¥¥¥¥¥¥¥¥¥¥¥¥
p3 = 33333333333333333333333333333333333333333333333333333333333333
3333333333333333333333333333333333333333333333333333333333¥¥¥¥¥¥¥¥

If we memset(p4, '4', 376), we have:
p4 = 44444444444444444444444444444444444444444444444444444444444444
4444444444444444444444444444444444444444444444444444444444444444444
4444444444444444444444444444444444444444444444444444444444444444444
4444444444444444444444444444444444444444444444444444444444444444444
4444444444444444444444444444444444444444444444444444444444444444444
4444444444444444444444444444444444444444444444¥¥¥¥¥¥¥¥
p3 = 33333333333333333333333333333333333333333333333333333333333333
3333333333333333333333333333333333333333333333333333333333¥¥¥¥¥¥¥¥

And if we then memset(p3, '3', 80), we have:
p4 = 44444444444444444444444444444444444444444444444444444444444444
4444444444444444444444444444444444444444444444444444444444444444444
4444444444444444444444444444444444444444444444444444444444444444444
4444444444444444444444444444444444444444444444444444444444444444444
4444444444444444444444444444444444444444444444444444444444444444444
4444444444444444444444444444444444444444444444¥¥¥¥¥¥¥¥
p3 = 33333333333333333333333333333333333333333333333333333333333333
3333333333333333333333333333333333333333333333333333333333¥¥¥¥¥¥¥¥
\end{minted}

\subsection{overlapping\_chunks\_2}

攻击失败,原因同上:
\begin{minted}[breaklines, frame=lines]{bash}
➜ ./overlapping_chunks_2
Now we allocate 5 chunks on the heap

chunk p1: 0x7f160857f000 ~ 0x7f160857f010
chunk p2: 0x7f1608580000 ~ 0x7f1608580080
chunk p3: 0x7f1608580080 ~ 0x7f1608580100
chunk p4: 0x7f1608580100 ~ 0x7f1608580180
chunk p5: 0x7f160857f010 ~ 0x7f160857f020

Let's free the chunk p4

Emulating an overflow that can overwrite the size of chunk p2 with (size of chunk_p2 + size of chunk_p3)

Allocating a new chunk 6: 0x7f1608581000 ~ 0x7f16085811c0

Now p6 and p3 are overlapping, if we memset(p6, 'B', 0xd0)
p3 before = AAAAAAAAAAAAAAAAAAAAAAAAAAAAAAAAAAAAAAAAAAAAAAAAAAAAAAAA
AAAAAAAAAAAAAAAAAAAAAAAAAAAAAAAAAAAAAAAAAAAAAAAAAAAAAAAAAAAAAAAAAAAA
AAAAZZZZZZZZZZZZZZZZZZZZZZZZZZZZZZZZZZZZZZZZZZZZZZZZZZZZZZZZZZZZZZZZ
ZZZZZZZZZZZZZZZZZZZZZZZZZZZZZZZZZZZZZZZZZZZZZZZZZZZZZZZZZZZZZZZZ
p3 after  = AAAAAAAAAAAAAAAAAAAAAAAAAAAAAAAAAAAAAAAAAAAAAAAAAAAAAAAA
AAAAAAAAAAAAAAAAAAAAAAAAAAAAAAAAAAAAAAAAAAAAAAAAAAAAAAAAAAAAAAAAAAAA
AAAAZZZZZZZZZZZZZZZZZZZZZZZZZZZZZZZZZZZZZZZZZZZZZZZZZZZZZZZZZZZZZZZZ
ZZZZZZZZZZZZZZZZZZZZZZZZZZZZZZZZZZZZZZZZZZZZZZZZZZZZZZZZZZZZZZZZ
\end{minted}

\newpage